%%% EDITION: 2nd Printing

\chapter{Gear}
\label{cha:gear}

The accelerated technological levels of \emph{Eclipse Phase} enable a number of devices for personal enhancement, survival, and other uses.


\section{Equipment rules}
\label{sec:equipment-rules}

The following rules apply to all technological items in \emph{Eclipse Phase}.


\subsection{Acquiring gear}
\label{sec:acquiring-gear}

During character creation, players purchase gear for their characters using the credits they have during the character creation process. Once play begins, however, characters must obtain any equipment they need the usual way: by buying, borrowing, making, or stealing it.

In the inner system, hypercorp, and Jovian Republic settlements - and other places where capitalism still reigns - gear acquisition is simply a matter of finding a seller and buying it. Each item has a listed cost, from Trivial to Expensive, as noted on the Gear Costs table. Due to local availability of resources, supply and demand, and legalities, these listed costs are meant to be approximations. When no other factors apply, the listed Average Cost for that category can be used. Otherwise the gamemaster should modify the item’s worth as they see fit, according to local economic factors, while still keeping it within that cost category range. The Cost Modifiers table lists out some suggested changes to an item’s cost, but these are simply recommendations, and can be ignored or followed as the gamemaster deems fit. The exact local conditions are largely up to the gamemaster to determine, as best fits their game.

In some circumstances, characters may attempt to haggle over gear prices. This is best handled as roleplaying, but the gamemaster may also call for an Opposed Persuasion Test (or possibly an Intimidation Test). The character who wins may increase or reduce the price by 10\% per 10 points of MoS.

In the outer system, anarchist, Titanian, scum, and other habitats that use the reputation economy, characters must rely on their rep scores to acquire the goods and services they need. The mechanics for this are covered under \emph{Reputation and Social Networks}, p. 285.

Characters are of course free to get their hands on equipment by any other means they devise - con schemes, borrowing from friends, and outright theft, with all of the appropriate tests and consequences. In some cases, acquiring gear may be an adventure unto itself.

\begin{table}
\begin{tabular}{|l|r|r|}
\hline
\multicolumn{3}{|c|}{\textbf{Gear costs}}			\\
\hline
\textbf{Category}	& \textbf{Range (in credits)}	& \textbf{Average (in credits)} \\
\hline
Trivial			& 1-99					& 50 \\
\hline
Low				& 100-499					& 250 \\
\hline
Moderate			& 500-1499				& 1000 \\
\hline
High				& 1500-9999				& 5000 \\
\hline
Expensive			& 10000+					& 20000 \\
\hline
\end{tabular}
\label{tab:gear-costs}
\end{table}

\begin{table}
\begin{tabular}{|l|r|}
\hline
\multicolumn{2}{|c|}{\textbf{Gear cost modifiers}}			\\
\hline
\textbf{Economic factor}	& \textbf{Suggested cost modifier} \\
\hline
Item Stolen			& -50\% \\
\hline
Item Used				& -25\% \\
\hline
Item Restricted		& +25\% \\
\hline
Item Illegal			& +50\% \\
\hline
Item Scarce			& +25\% \\
\hline
Item Extremely Rare		& +50\% \\
\hline
Item Common			& -25\% \\
\hline
\end{tabular}
\label{tab:gear-cost-modifiers}
\end{table}

\subsubsection{Fabricating gear}

Thanks to nanofabrication technology, characters may also create their own equipment using cornucopia machines and similar nanofab devices (p. 327). The character must have the appropriate blueprints to do so, whether they come with the fabber, are bought legitimately or on the black market, acquired with rep, or found online. Characters may also code their own blueprint desires, using the Programming: Nanofabrication skill.


\subsection{Gear modifiers}
\label{sec:gear-modifiers}

In the technological future, gear is a necessity. In many cases, use of equipment provides no bonuses, it simply allows a character to perform a task they would otherwise be unable to do. For example, it is impossible to pick a mechanical lock without lockpick or some sort of tool.

In other cases, however, gear provides a bonus to the task at hand. Climbing a wall may be possible without tools, but if you happen to have gecko gloves or other climbing gear, it’s going to be a lot easier. The specific modifier applied is usually noted in the gear item’s description, typically ranging from +10 to +30.

\subsubsection{Gear quality}

In both of the situations above, it is possible to have items that are of either exceptional or inferior quality, with corresponding positive or negative modifiers. The gear may be well-crafted, state-of-the-art, cutting-edge experimental, or simply top-of-the-line, applying an additional +10 to +30. Or it may be outdated, shoddy, or in disrepair, inflicting a -10 to -30 modifier (in some cases canceling out the basic gear bonus).

\subsubsection{Gear sizes}

On occasion, you’ll need to know how small or large a certain piece of equipment is. Though this is largely something the gamemaster can wing on the fly using common sense, we’ve listed sizes for many gear items that are unusual or so futuristic that the average player may not have a feel for what dimensions the tech likely is. These size categories are listed on the Gear Sizes table (p. 297). These sizes should be considered approximations, as depending on the manufacturer and process, some items may be smaller or larger than similar items. It is also important to keep in mind that as technology advances, the size and components of various equipment items shrink, so when in doubt, go with smaller.


\begin{table}
\begin{tabularx}{\textwidth}{|l|X|}
\hline
\multicolumn{2}{|c|}{\textbf{Gear sizes}}			\\
\hline
\textbf{Size category}	& \textbf{General dimensions and notes} \\ Nano					& So small that the item cannot be seen without the aid of a microscope or nanoscopic vision (p. 311), and may not be manipulated without fractal digits (p. 311) or similar tools. \\
\hline
Micro				& Anything ranging from the size of a barely visible small dot to an average insect. \\
\hline
Mini					& Mini items may be concealed within someone’s palm or small pockets. \\
\hline
Small				& Small items may be held in one hand and concealed in normal pockets.\\
\hline
Medium				& Medium size items are cumbersome to hold with one hand, ranging from the size of a 2-liter bottle to the size of a medium dog. They do not fit in pockets, but they may be concealed by larger coverings. \\
\hline
Large				& Roughly human-sized. \\
\hline
Huge					& Vehicles and other more massive objects. \\
\hline
\end{tabularx}
\label{tab:gear-sizes}
\end{table}


\subsubsection{Mass and encumbrance}

A character who is carrying too much gear should be slowed down, suffering negative modifiers both to their movement rates and their skill tests. Rather than micromanaging the weights of individual pieces of equipment, however, this matter is largely left to the gamemaster’s discretion, using common sense. If a character loads up beyond reason, apply modifiers as seem appropriate. The gamemaster should, however, keep in mind that many of the manufacturing materials used in \emph{Eclipse Phase} allow for items that are much lighter than current standards without any loss of durability or function (see \emph{Future Materials}, p. 298). Likewise, characters in low or microgravity environments can carry much larger loads.

\subsubsection{Concealing gear}

Characters may attempt to conceal items on their person, hoping at least to hide them from casual notice if not an intensive search. To determine how effectively the character conceals the equipment, make a Palming Test and note the MoS (the gamemaster may wish to roll this secretly). Whenever another character has a chance to notice the concealed item, they must succeed in a Perception Test and achieve a higher MoS than was scored on the Palming Test. The gamemaster should apply modifiers to both tests as appropriate. For example, concealing a large item like a sword would be difficult (-30), whereas wearing concealing clothing like a longcoat or multi-pocketed jumpsuit would help (+20). Likewise, a character who is not actively looking is less likely to notice the hidden gear (-30), whereas someone who conducts a physical search (+30) or who has enhanced vision to pierce protective layers will fare better.


\subsection{Design and fashion}
\label{sec:design-fashion}

Many objects in \emph{Eclipse Phase} closely resemble their early 21st century equivalents - a bottle of soda is still a transparent container holding a brightly colored liquid, clothing is obviously something you wear, and a knife still consists of a blade and a handle. The materials, processes, and mindsets that go into making them, however, are quite different. To start, very few items look have a uniform, mass-produced look, even if they were. The procedures of minifacturing and nanofabrication allow every individual item to be manufactured with a unique (or at least different) look. In areas with anarchist/reputation economies, in fact, where personal possessions have very little intrinsic value, expression and creativity are favored and so many items are artistically personalized (and actual hand-crafted items are rare and prized). Likewise, almost all equipment is designed with ergonomics and ease-of-use prioritized, so gear with soft curves, pleasing colors, and form-fitting shapes are common. Many items of personal technology, such as flashlights or small tools, are made in the form of ovoids that fit comfortably in the user’s hand or in similar forms that can be easily worn or attached to clothing. To someone from the 20th century, many common devices look like oddly colored rocks or decorative pieces of plastic or ceramic (in fact, many such items are referred to as ``blobjects'' by older transhumans).

The materials used to create everyday items are also advanced, ranging from aerogel and graphene to smart materials (p. 298) and exotic metamaterials with unusual physical properties. In practice, this means that most items are light, durable (with both tensile strength and/or flexibility, as needed), waterproof, dirt-repellent, and self-cleaning. Most gear is also designed with zero-G or microgravity functionality in mind, and can easily be clipped, tethered, or stuck to a surface with grip pads.

Almost all gear available in \emph{Eclipse Phase} is also available in forms that are wearable/usable by uplifted animals and non-humanoid morphs, such as novacrabs, slitheroids, and so on. Even if such customized gear is not immediately available, it is usually not difficult to nanofabricate. Smart materials (p. 298) also make interoperability between different morphs easy.

\subsubsection{Interface}

It is not uncommon for everyday devices to have no visible controls as they are designed to be operated via radio broadcasts from the user’s ecto or mesh inserts. Any items crafted for use in emergency, combat, survival, or exploration situations, however, will feature basic physical controls, just in case. Physical interfaces are typically controlled by touch pads that are nothing more than colored spots on the device’s surface, though some may also project a holographic interface display. Most equipment of this sort can can also be voice-activated and controlled.

Almost all devices are loaded with a complete set of help files and tutorials. Most electronics are also mesh-capable and equipped with specialized AIs (see \emph{Meshed Gear}, next page).

\subsubsection{Smart materials}

Many common items of technology are made from so-called smart materials. These devices contain - or sometimes consist entirely of - many small nanomachines that can both move and reshape themselves to alter the object’s shape, color, and texture. For example, smart clothing can transform from a suit of specialized cold weather clothing suitable for the Martian poles in winter to a fashionable suit in the latest style due to hundreds of thousands of tiny nanomachines in the clothing that shift and move to reshape the garment. Similarly, a tool made of smart materials can switch from a powered screwdriver to a wrench or a hammer, as the nanomachines move around and completely reshape the tool. Smart materials all contain specialized advanced nanomachine generators (p. 328) that keep them in perfect repair as long as they are regularly recharged.


\subsection{Future materials}
\label{sec:future-materials}

Many materials are available and commonly used in \emph{Eclipse Phase} that are rare, theorized, or unheardof today. The following entries note some of the more interesting.

\subsubsection{Aerogel}

Low-density, solid-state ``Frozen smoke'' is made by carefully foaming various materials, typically glasses or ceramics, to an ultra-low density state. Aerogel is semi-transparent and light-weight, feels like styrofoam, but acts as an incredible insulator against heat and cold. It is commonly used in habitats.

\subsubsection{Diamond}

Artificial diamond is lightweight and super-strong, has an extremely high melting point, and has nearperfect thermal conductivity. This makes it an ideal substance for hardening coated surfaces (armor) and creating super-tough diamond machinery.

\subsubsection{Fullerenes/Fullerites}

Fullerenes are molecular carbon structures (known as buckyballs, carbon nanotubes, and graphene) that are extremely strong (vastly stronger by weight than steel), heat-resistant, and can be either insulative or superconductive. This makes them useful in equipment as diverse as armor, electronics, sensor systems, or the cables of space elevators.

\subsubsection{Metallic foam}

Metal foam is created by adding foaming agents to liquid metals, resulting in extremely lightweight metallic structures - light enough to float on water. Ideal for habitat construction and floating cities.

\subsubsection{Metallic glass}

Metallic glass are metals highly alloyed to possess a disordered (rather than crystalline) atomic structure with unique combinations of stiffness and strength, making it a good wear surface and alternative to ceramics in armor. It is also useful for its unusual (for a metal) electrical resistance properties.

\subsubsection{Metamaterials}

Metamaterials have unusual physical properties (usually electromagnetic) due to their structure, such as having a negative refractive index. Metamaterials are used to create invisibility cloaks (p. 316), superlenses, phased array optics, and impressive 2-D holograms.

\subsubsection{Refactory metals}

These metallic alloys have extremely high melting points, making them ideal for extremely hot engine systems, atmospheric entry vehicles, and hypersonic craft.

\subsubsection{Transparent alumina}

In transparent form, this ceramic is often known as sapphire. Transparent alumina is harder than steel and zero-g casting techniques allow for intriguing transparent construction designs, so long as its poor tensile strength is respected.


\section{Meshed gear}
\label{sec:meshed-gear}

Almost all technology in \emph{Eclipse Phase} is designed to be operated via radio signals from the user’s basic implant, although models usable by characters without basic implants are also available. In addition all devices contain a nearly microscopic computer and radio link (known as a ``voice”) that allows the user to easily locate the object and that reports on the condition of the object or device, how to properly use and care for it, as well as telling the user when it needs to be repaired and how. Most are discrete and highly useful, but cheaply made goods sometimes have overly annoying voices.

This means that almost all devices can be accessed via the mesh or directly if within radio range. This makes them vulnerable to hacking and intrusion attempts (p. 254) as well as radio jamming (p. 262). Many devices are, however, publicly accessible (see \emph{Spimes}, p. 238). Meshed gear may also be tracked through the mesh (p. 251). For privacy and security, these devices are often slaved to other systems (see \emph{Slaving Devices}, p. 248); devices worn/carried by characters are usually made part of the personal area network and slaved to the character’s mesh inserts/ ecto. For more info on meshed devices, see the \emph{Mesh chapter}, p. 234.

Many devices come equipped with AIs, who are equipped with skillsofts that enable them to operate the device on their own, as according to voiced instructions or commands issued through the net. AIs are described on p. 264 and p. 331.


\subsection{Radio and sensor ranges}
\label{sec:radio-sensor-ranges}

In \emph{Eclipse Phase}, almost all devices are equipped with small radios so that they may be meshed. Likewise, many pieces of gear are equipped with sensors such as cameras, microphones, or other detectors. The Radio and Sensor Ranges table notes what range these devices operate at.

\begin{table}
\begin{tabularx}{\textwidth}{|l|l|l|X|}
\hline
\multicolumn{4}{|c|}{\textbf{Radio and sensor ranges}}			\\
\hline
\textbf{Size category}	& \textbf{Urban range}	& \textbf{Urban range}	& \textbf{Examples} \\
\hline
Nano 				& 20 meters 			& 100 meters			& Smart Dust, Nanobot/Microbot Swarms \\ Micro				& 50 meters			& 500 meters			& Microbugs \\ Mini					& 1 kilometer			& 20 kilometers		& Mesh Inserts \\ Small				& 5 kilometers			& 50 kilometers		& Ectos, Miniature Radio Farcasters, Portable Sensors \\ Medium				& 25 kilometers		& 250 kilometers		& Radio Boosters, Vehicle Sensors \\ Large				& 500 kilometers		& 5000 kilometers		& \\
\hline
\end{tabularx}
\label{tab:radio-sensor-ranges}
\end{table}


\subsection{Power}
\label{sec:power}

All of the powered devices in \emph{Eclipse Phase} require electricity to function. With rare exceptions, most of them rely on either solar cells or powerful batteries. These batteries are high-density, room-temperature superconductors with 25 times the capacity of the best batteries in common use in the early 21st century. Such batteries may also be constructed so that they are flexible, printed on devices, or woven into fabric. They are good for 100-500 hours of use, and will alert the user when they start running low. More powerful radio-isotope nuclear batteries are also available, heavily shielded so they do not emit radiation and good for 3 years or more of use.

In short, power should rarely be an issue in \emph{Eclipse Phase} games, unless it happens to fit the plot. Power failure could also result from a critical failure roll.


\section{Personal augmentation}
\label{sec:personal-augmentation}

Almost all citizens of the solar system, whether human, AI, or uplifted animal, use various forms of biological, cybernetic, or nanotechnological augmentation. The following is a list of the most common types.

Unless otherwise noted, any bonuses from personal augmentations are both compatible and cumulative with bonuses from other enhancements.


\subsection{Standard augmentations}
\label{sec:std-augmentations}

Most morphs produced in the solar system include the following augmentations.

\subsubsection{Basic biomods}

Almost universal in biomorphs, many habitats will not allow individuals to visit/immigrate if their biomorph does not possess these biomods in order to preserve public health. Basic biomods consists of a series of genetic tweaks, tailored virii, and bacteria that speed healing, greatly increase disease resistance, and impede aging. A morph with basic biomods heals twice as fast as an early 21st century human, gradually regrows lost body parts, is immune to all normal diseases (from cancer to the flu), and is largely immune to aging. In addition, the morph requires no more than 3-4 hours of sleep per night, is immune to ill-effects from longterm exposure to low or zero gravity, and does not naturally suffer from biological problems like depression, shock reactions after being injured, or allergies. \textbf{[Moderate, but included for free in most biomorphs]}

\subsubsection{Basic mesh inserts}

Mesh inserts are ubiquitous among modern morphs. This network of cybernetic brain implants is essential equipment for anyone who wants to stay connected and make full use of the wireless mesh. The interconnected components of this system include:

\begin{itemize}
\item \textbf{Cranial computer:} This computer serves as the hub for the character’s personal area network and is home to their muse (p. 264). It has all of the functions of a smartphone and PDA, acting as a media player, meshbrowser, alarm clock/calendar, positioning and map system, address book, advanced calculator, file storage system, search engine, social networking client, messaging program, and note pad. It manages the user’s augmented reality input and can run any software the character desires (see \emph{Software}, p. 331). It also processes XP data, allowing the user to experience other people’s recorded memories, and also allowing the user to share their own XP sensory input with others in real-time. Facial/image recognition and encryption software (p. 331) are included by default.
\item \textbf{Radio tranciever:} This transceiver connects the user to the mesh and other characters/devices within range. It has an effective range of 20 kilometers in deep space or other locations far from radio interference and 1 kilometer in crowded habitats.
\item \textbf{Medical sensors:} This array of implants monitors the user’s medical status, including heart rate, respiration, blood pressure, temperature, neural activity, and much more. A sophisticated medical diagnostic system interprets the data and warns the user of any concerns or dangers.
\end{itemize}

Using any of these functions is as easy as thinking. \textbf{[Moderate, but included for free in most morphs]}

\subsubsection{Cortical stack} A cortical stack is a tiny cyberware data storage unit protected within a synthdiamond case the size of a grape, implanted at the base of the skull where the brain stem and spinal cord connect. It contains a digital backup of that character’s ego. Part nanoware, the implant maintains a network of nanobots that monitor synaptic connections and brain architecture, noting any changes and updating the ego backup in real time, right up to the moment of death. If the character dies, the cortical stack can be recovered and they may be restored from the backup (see Resleeving, p. 271). Cortical stacks do not have external or wireless access (for security), they must be surgically removed (see Retrieving a Cortical Stack, p. 268). Cortical stacks are extremely durable, requiring special effort to damage or destroy. They are commonly recovered from bodies that have otherwise been pulped or mangled. Cortical stacks are intentionally isolated from mesh inserts and other implants, as a security measure to prevent hacking or external tampering. \textbf{[Moderate, but included for free with most morphs]}

\subsubsection{Cyberbrain}

Cybernetic brains are where the ego (or controlling AI) resides in synthmorphs and pods. Modeled on biological brains, cyberbrains have a holistic architecture and serve as the command node and central processing point for sensory input and decision-making. Only one ego or AI may ``inhabit'' a cyberbrain at a time; to accommodate extras, mesh inserts (p. 300) or a ghostrider module (p. 307) must be used. Since cyberbrains store memories digitally, they have the equivalent of mnemonic augmentation (p. 307). They also have a built-in puppet sock (p. 307) may be remote-controlled, though this option may be removed by those who value their security. Cyberbrains are vulnerable to brainhacking (p. 261) and other forms of electronic infiltration/attack. Cyberbrains come equipped with two or more pairs of external access jacks (p. 306), usually located at the base of the skull, which allow for direct wired connections. \textbf{[Moderate, but included for free in all synthetic morphs and pods]}

\subsection{Bioware}
\label{sec:bioware}

Bioware augmentations can be acquired either as a genemod when the morph is designed and grown or as a later modification to an existing morph, either by using nanomachines to modify the morph’s tissue or by externally growing the organ and implanting it. Bioware may be used to enhance biomorphs (including pods and uplifts), but not synthmorphs. Bioware may be used to enhance biomorphs (including pods and uplifts), but not synthmorphs (see Synthmorphs and Bioware, p. 306).

\subsubsection{Enhanced senses}

The following are a list of the most common enhanced senses. Each is also available as a cybernetic implant, but bioware is much more common.

\textbf{Direction Sense:} The character has an innate sense of direction and distance using advanced inertial navigation. The character can arbitrarily define any point as ``north'' and keep track of which direction that is, as well as knowing approximately how far they have come. Characters with this augmentation can always retrace any route they have taken, only experiencing difficulty with three-dimensional routes lacking navigational markers (such as deep space or undersea; apply a -30 modifier). Since positioning inside habitats by anyone with basic mesh inserts is an automatic affair, only characters venturing to remote locations require this augmentation. \textbf{[Low]}

\textbf{Echolocation:} The character possesses sonar similar to that of a bat or dolphin. The character bounces brief ultrasonic pulses off their surroundings and uses them to form an image of these surroundings through the pattern of refl ections of these pulses received by the character’s ears. For more details, see Using Enhanced Senses, p. 302. This augmentation works in both air and water and has a range of 20 meters in air and 100 meters in water. \textbf{[Low]}

\textbf{Enhanced Hearing:} The morph’s ears are enhanced to hear both higher and lower frequency sounds - the range of sounds they can hear is twice that of normal human ears (see Using Enhanced Senses, p. 302). In addition, their hearing is considerably more sensitive, allowing them to hear sounds as if they were five times closer than they are. A character with this augmentation can easily overhear even a softly spoken conversation at another table in a small restaurant. This augmentation provides a +20 modifier to all Perception Tests involving hearing. \textbf{[Low]}

\textbf{Enhanced Smell:} The morph’s sense of smell is equal to that of a bloodhound. The user can identify both chemicals and individuals by smell, and can track people and chemically reactive objects by smell as long as the trail was made within the last several hours and has not been obscured. The character can also gain a general sense of the emotions and health of any character within 5 meters (+20 to Perception or Kinesics Tests to do so). \textbf{[Low]}

\textbf{Enhanced Vision:} The morph’s eyes have tetrachromatic vision capable of exceptional color differentiation. These eyes can also see the electromagnetic spectrum from terahertz wave frequencies to gamma rays, enabling them to see a total of several dozen colors, instead of the seven ordinary human eyes can perceive. In addition, these eyes have a variable focus equivalent to 5 power magnifiers or binoculars. This augmentation provides a +20 modifier to all Perception Tests involving vision. For further applications, see Using Enhances Senses, p. 302. \textbf{[Low]}

\subsubsection{Mental augmentations}

Mental augmentations are extremely common.

\textbf{Eidetic Memory:} The character can remember everything that ever happened to them, in detail, with no long term memory loss. For example, they can recite a page they read in a book a month ago, recall a string of 200 random characters they viewed a year ago, or even tell you what they had for breakfast on a particular date a decade ago. However, they can only remember things they paid attention to. The character will not remember the contents of a note on someone’s desk if they merely glanced at it; they must specifically have read it. No effort is required to use this augmentation, the character merely needs to attempt to remember a specific fact. \textbf{[Low]}

\textbf{Hyper Linguist:} The morph’s brain maintains the linguistic flexibility of a small child, allowing the character to learn languages with great ease. This functions as the Hyper Linguist trait, p. 146. \textbf{[Low]}

\textbf{Math Boost:} This implants functions as the Math Wiz trait, p. 146. \textbf{[Low]}

\textbf{Multiple Personalities:} The character’s brain is intentionally partitioned to accommodate an extra personality. This multiplicity is not viewed as a disorder, but as a cognitive tool to help people deal with their hypercomplex environments. This extra personality can be an NPC run by the gamemaster, a separate character (in ego form only) made by the player, or the downloaded fork of another character. For all intents and purposes, the extra personality is treated as a separate ego (i.e., it may fork separately), except that both personalities are backed up in the same cortical stack and if downloaded they must be placed in separate morphs or in another morph with this implant.

Only one ego may be in control of the morph at a time. The other resides in the background, still active, but not on a surface level. Each ego is completely aware of what the other is doing, thinking, etc. If for some reason the subsumed personality wants to come to the fore, but the other personality won’t relinquish control, make an Opposed WIL x 3 Test. Each ego has its own Lucidity and Trauma Threshold, and they track stress and trauma separately. Any psi attacks or social/ mental influences only affect the personality at the fore. Having an extra ego in your head, working in the background, is helpful for multitasking. The character receives an extra Complex Action each turn that may only be used for mental or mesh actions. \textbf{[High]}

\subsubsection{Physical augmentations}

Most physical bioware augmentations are derived from the capabilities of animals.

\textbf{Adrenal Boost:} This adrenal gland enhancement supercharges the character’s adrenal response to situations that invoke stress, pain, or strong emotions (fear, anger, lust, hate). When activated, the concentrated burst of norepinephrine accelerates heart rate and blood flow and burns carbohydrates. In game terms, this allows the character to ignore the -10 modifier from 1 wound and temporarily increases REF by +10 (also boosting REF-linked skills and Initiative). These modifiers apply until the character has calmed down (if the character also has endocrine control, p. 304, then adrenal boosts can be activated and deactivated at will, and the negated wounds are cumulative). \textbf{[High]}

\textbf{Bioweave Armor (Light):} Bioweave armor involves lacing the morph’s skin with artificial spider silk biological fibers. This provides an Armor rating of 2/3 without changing the appearance, texture, or sensitivity of the morph’s skin. This armor is cumulative with worn armor. \textbf{[Low]}

\textbf{Bioweave Armor (Heavy):} Heavy bioweave armor involves lacing the morph’s skin with a denser and thicker network of the same fibers. The morph’s skin becomes thicker and somewhat less flexible except at the joints. The morph’s skin also has an unusually smooth look, and a distinctively smooth and tough-feeling texture. This provides an Armor rating of 3/4 without decreasing the morph’s mobility. The character’s sense of touch, however, is significantly reduced (-20 modifier) except on their hands, feet, and face. This armor is cumulative with worn armor. \textbf{[Moderate]}

\textbf{Carapace Armor:} Carapace armor combines bioweave armor with hard but flexible plates of a chitin-ceramic hybrid material modeled on the microscopic structure and texture of arthropod exoskeletons. This armor is obvious and has a somewhat crocodilian or insectoid appearance (character’s choice). The morph is completely hairless as well. This provides an Armor rating of 11/11. This armor is not cumulative with worn armor. \textbf{[Moderate]}

\textbf{Chameleon Skin:} The morph’s skin is augmented with complex chromatophores so that it changes color like the skin of a chameleon or an octopus. The morph can match the appearance of almost any color and most patterns. This provides a +20 modifier to Infiltration Tests to avoid being seen or noticed, as long as the character is stationary or not moving faster than a slow walk. The character must be nude or wearing smart clothing (p. 325) of the same color/pattern. If incompletely camouflaged, or if moving faster, reduce the modifier to +10. In addition to blending in, the character can also consciously change the color and pattern of their skin to deliberately stand out (+20 on Perception Tests to notice) or simply to produce attractive or interesting colors or patterns. \textbf{[Low]}

\textbf{Circadian Regulation:} The morph only requires 2 hours of sleep to maintain health and function at peak mental capacity. The character dreams constantly while asleep and can both fall asleep and wake up almost instantly. In addition, the character can easily and with no ill-effects shift to a 2-day cycle, where they are awake for 44 hours and sleep for 4. \textbf{[Moderate]}

\textbf{Claws:} The morph has retractable claws like those of a cat. These claws do not interfere with the character’s manual dexterity and are razor sharp. However, they are relatively small and only do 1d10 + 1 + (SOM $\div$ 10) damage, with an AP of -1. As a result, they are legal in almost all habitats and are considered tools as much as weapons. \textbf{[Low]}

\textbf{Clean Metabolism:} The morph’s symbiotic bacteria, gut flora, and glands have been genetically engineered to keep the morph ``clean.'' The morph also produces smart antibiotics that prevent the growth of any bacteria or yeasts in it or on its skin. As a result, the morph is completely immune to infections, dental cavities, and bad breath, its sweat has no scent, and the morph’s efficient digestion produces somewhat less solid waste and less odorous chemicals. \textbf{[Moderate]}

\textbf{Drug Glands:} The morph has specially-tailored glands designed to produce specific hormones or chemicals and release them in the body. The character has control over these glands and can release the chemicals at will. Each type of drug gland is considered a separate enhancement. For potential drugs and chemicals, see p. 317. \textbf{[One Cost Category Higher Than Drug Cost]}

\textbf{Eelware:} Derived from electric eel genetics, a character can have eelware implanted so that it connects to a network of bioconductors in the hands and feet (or other limbs), allowing the character to generate stunning shocks with a touch. Eelware inflicts shock damage (p. 204) exactly like a pair of shock gloves. Eelware can also be used to power implants and specially designed handheld devices by touch. \textbf{[Low]}

\textbf{Emotional Dampers:} This low-cost alternative to endocrine control (p. 304) allows the user to voluntarily damp their morph’s emotional responses and various non-verbal cues like pupil dilation, eye movement, or vocal tone. Using this augmentation allows the user to lie and conceal their emotions in such as way as oo fool the keenest observer; apply a +30 modifier to Deception and Impersonation Tests. This modification does not affect methods of detecting lies and emotions that involve reading the character’s neural state, including psi-gamma sleights. However, this augmentation damps out all emotional responses and so causes the character to be less persuasive in real- time personal interactions, imposing a -10 modifier to other Social skill tests like Persuasion. Characters can turn this augmentation on or off at will. \textbf{[Low]}

\textbf{Endocrine Control:} This augmentation modifies the morph’s endocrine system, giving the character fine control over their hormone output. This allows the character to completely control their appetite and emotions and to regulate pain. They receive a +30 modifier against the effects of hunger, fear, and any forms of emotional manipulation, such as the Drive Emotion sleight. This augmentation also allows character to lie with perfect conviction and to completely fool all methods of lie detection that do not rely on the target’s neural output; apply a +20 modifier to Deception Tests. It also allows the character to remain awake for 48 hours without penalty, but after this time the character begins experiencing normal fatigue. Finally, the ability to regulate pain reception allows the character to ignore the -10 modifier from 1 wound. \textbf{[High]}

\textbf{Enhanced Pheromones: }The morph’s biochemistry has been altered so that it produces enhanced pheromonal signals that subconsciously affect the behavior of other humans in the vicinity. These pheromones make the character more attractive and trustworthy to the target; apply a +10 modifier to appropriate Social skill tests, such as Persuasion. This augmentation only affects characters who can smell the pheromones, and it does not affect uplifts or xenomorphs. [Low] Enhanced Respiration: By boosting both lung efficiency and the blood’s oxygen-carrying capacity, the character can live comfortably in both high and low pressure environments, from 0.2 atmospheres to 5 atmospheres, with no dizziness or need for gradual decompression. In addition, the character can hold their breath for up to 30 minutes when performing minimal activity or for up to 10 minutes while performing highly strenuous activity. \textbf{[Low]}

\textbf{Gills:} The morph’s lung tissue has been adapted to function as gills, allowing the morph to breathe both air and water, as long as the water is not toxic or too stagnant. Characters with this augmentation breathe in water and then expel the water through slits just underneath their lowest pair of ribs that seal when the character is not underwater. \textbf{[Low]}

\textbf{Grip Pads:} The morph possesses specialized pads on its palms, lower arms, shins, and the bottoms of its feet. Designed to emulate the pads on gecko feet, characters can support themselves on a wall or ceiling by placing any two of these pads against any surface not made from a material specially designed to resist this augmentation. Characters can climb any surface and move easily across ceilings that can support their weight. Apply a +30 modifier to Climbing Tests. The pads must be free to touch the surface the character is climbing (no gloves). The nature of these pads is obvious to anyone looking at them, but they do not impair the character’s sense of touch or manual dexterity. If combined with the vacuum sealing augmentation, the character can even stick to surfaces in the vacuum of space. \textbf{[Low]}

\textbf{Hibernation:} The character can voluntarily reduce the morph’s metabolism to the point that the morph requires only 5\% of the normal amount of food, water, and air. The character appears to sink into a deep sleep, but can maintain a dim awareness of both touch and sound and so can be easily awakened. Entering or leaving this state requires 3 minutes where the character is relatively helpless. With sufficient air, characters can safely hibernate for up to 40 days without food or water. \textbf{[Low]}

\textbf{Muscle Augmentation:} The morph’s muscle mass has been enhanced and toned and myofibers strengthened. Apply a +5 modifier to SOM. \textbf{[High]}

\textbf{Neurachem:} This bioware modification enhances the character’s chemical synapses and juices their neurotransmitters, drastically speeding up neural connections. Neurachem can be mentally activated or triggered by charged emotions. Level 1 neurachem increases the character’s Speed stat by +1, with no side effect. Level 2 raises the Speed stat by +2, but each time it is used the character suffers a nervous system fatigue hangover for 1 hour after the boost wears off (apply a -20 modifier to all actions). \textbf{[High (Level 1), Expensive (Level 2)]}

\textbf{Poison Gland:} Similar to the drug gland, this morph has special glands that produce poisons, like the venom glands of a snake. The morph has poison glands in its fingers and mouth, so that it can deliver either poison by scratching someone with a fingernail, biting them hard enough to draw blood, or even by sharing a beverage with someone or spitting into their drink. The morph is immune to the poisons it produces. These glands may not produce nanotoxins. \textbf{[Low]}

\textbf{Prehensile Feet:} The morph’s feet and leg joints are altered so that its toes are longer and more dexterous and the big toe is transformed into an opposable thumb. Physically, the morph’s feet resemble a longer narrower hand or a human foot with finger (and thumb)-like toes. The character can walk normally but must wear specially designed shoes. However, this morph runs somewhat slower than a morph with unmodified feet (-1 meter per Action Turn). In addition, the morph’s hips are slightly modified to allow greater mobility. In a properly constructed chair, or when floating in zero-G, the character can use both their hands and their feet to manipulate the same object. Most morphs used by characters who live in zero-G possess this augmentation. \textbf{[Low]}

\textbf{Prehensile Tail:} A long (1.5 meters) prehensile tail is added to the morph’s backside, extending out from the tailbone. This tail is prehensile and may be used to grab, hold, and even manipulate objects. The character can control the tail’s movements with concentration, but it otherwise tends to move on its own. The tail also improves the character’s balance; apply a +10 to any Physical skill tests where balance is a factor. \textbf{[Low]}

\textbf{Sex Switch:} A complex suite of alterations allows the character to switch their physical sex to male, female, hermaphrodite, or neuter. This change is mentally triggered but takes approximately 1 week to complete. \textbf{[Moderate]}

\textbf{Skin Pocket:} The morph has a pocket within its skin layer, capable of holding and providing concealment (+30) for small items. \textbf{[Trivial]}

\textbf{Temperature Tolerance:} The morph’s temperature regulation and circulation are both substantially enhanced allowing the character to survive in temperatures as low as -30 degrees Celsius and as high as 60 degrees Celsius without discomfort or ill effects. \textbf{[Low]}

\textbf{Toxin Filters:} The morph gains an improved liver and kidneys and biological filters in its lungs. Characters with this augmentation are immune to all chemical and biological toxins, including everything from recreational chemicals to nerve agents to spoiled food. In addition, the character can safely and comfortably breathe smoke and drink salt water. Unlike medichines, toxin immunity prevents the character from experiencing even brief harm or discomfort from a toxin (medichines merely rapidly repair damage caused by the toxin and then remove it from the morph). This augmentation provides no resistance to concentrated acid, nanotechnological attacks, or similar destructive agents. Some characters with this augmentation learn to enjoy the taste of various chemical toxins like cyanide or arsenic. \textbf{[Moderate]}

\textbf{Vacuum Sealing:} To possess this augmentation, the character must also possess some form of bioware armor or carapace armor. The morph has been specially designed to survive the effects of vacuum. The character’s skin resists vacuum as well as protecting the wearer from temperatures from -75 to 100 C. In addition, the character’s mouth, nose, and other orifices can seal sufficiently well to resist vacuum, and the morph possesses a special membrane that extends over their eyes, allowing the character to see in vacuum without risking any eye damage. This augmentation is usually combined with either the enhanced respiration or oxygen storage augmentation, or both together. \textbf{[High]}

\subsubsection{Synthmorphs and bioware}

Though bioware is preferred and more common, many types of bioware can be mimicked with cybernetics. This is especially useful for synthmorphs/ robots, which cannot be enhanced with bioware. The following bioware items may be replicated as cybernetics for synthmorphs and robots:

\begin{itemize}
	\item Chameleon Skin
	\item Drug Glands
	\item Eelware
	\item Emotional Dampers
	\item Enhanced Senses (All)
	\item Grip Pads
	\item Mental Augmentations (All)
	\item Muscle Augmentation
	\item Neurachem
	\item Poison Glands
	\item Prehensile Feet
	\item Prehensile Tail
\end{itemize}

\subsection{Cyberware}
\label{sec:cyberware}

Very little cyberware is physically implanted. Instead, the morph is placed in a healing vat (p. 326) and the vat’s nanobots construct the cyberware inside the biomorph’s body. Cyberware is rarely used for anything that can be accomplished using bioware.

Synthmorphs and bots may also also use cyberware.

\subsubsection{Enhanced senses}

In addition to being able to duplicate the affects of all bioware enhanced senses, there are a few enhanced senses that can only be produced using cyberware.

\textbf{Anti-Glare:} This visual mod eliminates penalties for glare. \textbf{[Low]}

\textbf{Electrical Sense:} The character can sense electric fields. Within 5 meters, the character can instantly tell if an electrical device is on or off and can see the precise location of electrical wiring behind a wall or inside a device. This sense gives the character a +10 modifier on any test involving analyzing, repairing, or modifying electrical equipment. \textbf{[Low]}

\textbf{Radiation sense:} The character can sense the presence and approximate source of all forms of dangerous radiation, including neutrons, charged particles, and cosmic rays. \textbf{[Low]}

\textbf{T-Ray Emitter:} Mounted under the skin of the user’s forehead, this implant generates low-powered beams of terahertz radiation (T-rays) that allow the character to see using reflected T-rays. As discussed in Using Enhanced Senses, p. 302, this implant combined with the enhanced vision enhancement (or a terahertz sensor) allows the user to effectively see through cloth, plastic, wood, masonry, composites, and ceramics as well as being able to determine the composition of various materials. This implant allows the user to see using reflected T-rays for 20 meters in a normal atmosphere and for 100 meters in vacuum. \textbf{[Low]}

\subsubsection{Mental augmentations}

These cybernetic augmentations enhance the brain and mental functions.

\textbf{Access Jacks:} Usually located in the base of the skull or neck, this implant is an external socket with a direct neural interface. It allows the character to establish a direct wired connection using a fiberoptic cable to external devices or other characters, which can be useful in places where wireless links are unreliable or complete privacy is required. Two characters linked via access jack can ``speak'' mind-to-mind and transfer information between their mesh inserts and other implants. All synthmorphs have these by default. \textbf{[Low]}

\textbf{Dead Switch:} This cortical stack (p. 300) accessory is designed to keep the stack from falling into the wrong hands. If the morph is killed, the dead switch wipes and melts the cortical stack completely, so that the ego cannot be recovered. This option is generally only used by covert operatives with recent backups. \textbf{[Low]}

\textbf{Emergency Farcaster:} Only characters with cortical stacks can possess this augmentation. The morph has an implanted quantum farcaster (p. 314) linked to a highly secure storage facility. The high cost of this implant also covers the cost of this storage. Using standard radio and quantum encryption, the farcaster broadcasts full backups of the character’s ego (pulled from the cortical stack) once every 48 hours. At the gamemaster’s discretion, the backup interval may be scheduled more or less frequently, keeping in mind that ego broadcasts are generally limited for security purposes and because they hog bandwidth. These broadcasts only work when the character is in radio contact with the storage facility and is typically only used inside a habitat to broadcast backups back to a nearby space ship. If the radio broadcasts are blocked or jammed, this device cannot make backups.

In the event of a farcaster failure, this augmentation also includes a single-use emergency neutrino broadcaster (p. 314) as well. This broadcaster contains approximately 10 nanograms of antimatter stored in an orange-sized triply-redundant magnetic containment vessel. If the character is dying or urgently wishes to depart the morph, this tiny amount of antimatter is brought into contact with a similarly tiny amount of matter in a controlled fashion that generates a single brief and carefully coded neutrino pulse of the ego’s most recent backup. However, the heat generated by this process literally cooks the entire morph, killing it and destroying all implants and electronics in or on it.

This entire process takes less than 0.1 second and the broadcast can be received as long as the neutrino receiver is within 100 astronomical units of the character. Within the solar system, this implant effectively guarantees the character’s backup. It is less useful on exoplanets where the character is out of neutrino range of their backup facility. The amount of antimatter carried by this implant is sufficiently small enough that it does not produce an explosion and will not damage any surrounding objects. Most habitats carefully scan all visitors to determine if they have this implant and if the amounts of antimatter involved are sufficiently low as not to pose a danger to the habitat and its inhabitants, and some ban this implant entirely. \textbf{[Expensive]}

\textbf{Ghostrider Module:} This implant allows the character to carry another infomorph inside their head. This infomorph could be another muse, an AI, a backed-up ego, or a fork. The module is linked to the character’s mesh inserts, so the ghost-rider can access the mesh. The character may limit the ghostrider’s access, or may allow them direct access to their sensory information, thoughts, communications, and other implants. \textbf{[Low]}

\textbf{Mnemonic Augmentation:} A character with this augmentation and a cortical stack can access digital recordings of all of the sensory data they have experienced in XP format (and they may share these recordings with others). Mnemonic augmentation differs from the eidetic memory bioware because it allows characters to digitally share all of their sensory data with others. It also allows them to closely examine sensory data they did not initially look at. For example, If the character glanced at a note but did not read it, they can later use image enhancement software to enhance this image and in most cases actually read what the note said. Mnemonic augmentation allows the character to clearly hear all background noises, like a conversation at a nearby table that the character only initially heard a few words of. Using mnemonic augmentation to retrieve a specific piece of information is quite easy, but usually requires between 2 and 20 minutes of concentration. \textbf{[Low]}

\textbf{Multi-Tasking:} Only characters with cortical stacks can possess this augmentation. The character has an advanced computer installed in their brain that uses the data in the cortical stack to create several simultaneous short-term forks to handle various mental tasks. By design, this computer automatically reintegrates all of these forks into the character’s core personality after a maximum of 4 hours, earlier if desired. This augmentation allows the character to both plan a speech and engage in intensive mesh-browsing while simultaneously fighting a gun battle or running from pursuit, since each of the forks operates independently. However, these forks can only perform purely mental or on-line interactions. This augmentation can produce a maximum of two forks at a time, giving the character an extra two Complex Actions on every Action Phase for mental or on-line actions. This implant cannot be used simultaneously with any other augmentation that allows for extra actions, or with the mental speed augmentation (p. 308). \textbf{[High]}

\textbf{Puppet Sock:} This implanted computer allows the biomorph’s body (the ``puppet”) to be controlled by another character (the ``puppeteer”). While active, the puppet has no control over their body and is simply along for the ride (at the gamemaster’s discretion, puppets who are tormented by repeated or extensive loss of control may suffer mental stress). The puppeteer may directly ``jam'' the puppet or remote control it in the same way that robots and pods are teleoperated (p. 196). The puppeteer must either be ghost-riding the puppet (see the Ghostrider Module, p. 307) or have a direct communications link (via mesh, radio, laser, etc.). \textbf{[Moderate]}

\subsubsection{Physical augmentations}

This implants enhance the morph’s physical body.

\textbf{Cyberclaws:} The bones on the back of the morph’s hand are bonded to smart material claws. These claws can extend through concealed ports in the morph’s skin and extend 6 inches past the morph’s knuckles. These razor-sharp weapons inflict 1d10 + 3 + (SOM $\div$ 10) damage and have an AP of -2. If combined with eelware (p. 304), they can also inflict electric shocks. Likewise, cyberclaws can also deliver poison or nanotoxins secreted from a poison gland (p. 305) or implanted nanotoxins. \textbf{[Low]}

\textbf{Cyberlimb:} In an age when arms and legs can easily be regrown, many people consider cybernetic prostheses to be vulgar and distasteful. The Scum and others, however, treat them as iconic symbols of self-expression. Standard replacement cyberlimbs function the same as their biological equivalents, though that particular limb receives a +3/+3 Armor bonus when targeted specifically (this bonus does not apply to synthmorphs). Cyberlimbs may be masked to look real (see Synthetic Mask, p. 311), and may also feature small compartments for hiding/storing small objects. \textbf{[Moderate]}

\textbf{Cyberlimb Plus:} More extravagant cyberlimb models are also available, though they require more severe body alteration to accommodate. These limbs apply a +5 SOM bonus per limb (maximum +10). They may be replacement limbs or ``extra'' limbs anchored in the body’s skeletal frame. These cyberlimbs may not be masked. \textbf{[High]}

\textbf{Hand Laser:} The morph has a weapon-grade laser implanted in its forearm, with a flexible waveguide leading to a lens located between the first two knuckles on the morph’s dominant hand. The laser fires from this waveguide, inflicting 2d10 damage with 0 AP. The laser is powered by a small nuclear battery located in the morph’s torso, good for 50 shots before it must be recharged like other beam weapon batteries (p. 338). \textbf{[Moderate]}

\textbf{Hardened Skeleton:} The morph’s skeleton has been laced with strengthening materials. Apply a +5 DUR and +5 SOM bonus. \textbf{[High]}

\textbf{Oxygen Reserve:} The morph has a miniature oxygen tank and rebreather installed in its torso. This implant provides the equivalent of the life support system in a light vacsuit (p. 333), allowing the character to breathe comfortably for up to 3 hours. It feeds oxygen directly to the morph’s blood stream, avoiding problems with pressure changes. Implanted sensors automatically cause the character to use the stored oxygen if they detect poisonous or insufficient atmosphere. Without vacuum sealing, the character can only survive in vacuum for 5 minutes, but remains conscious and active for the entire time, giving them far more time to find shelter or a vacsuit than characters without this implant. For every hour the character is in a breathable atmosphere, this implant recovers one hour of oxygen storage. The implant can be fully recharged within 15 minutes if the character is in a high-pressure mostly oxygen atmosphere. \textbf{[Low]}

\textbf{Reflex Boosters:} The morph’s spinal column and nervous system is rewired with superconducting materials, boosting transmission speed. This raises the character’s REF by +10 and improves Speed by +1. \textbf{[Expensive]}

\subsection{Using enhanced senses}
\label{sec:using-enhanced-senses}

Personal augmentations and technological aids have drastically increased the sensory capabilities of most transhumans. The following notes provide some details on what capabilities these sensory functions provide. The capabilities are typically the same whether it’s a biological sense or a technological sensor, though tech sensors can ``turn off'' certain wavelengths and sense only specific frequencies, whereas biological senses perceive the full spectrum with no ability to filter parts out.

\subsubsection{Sensory databases}

Both technological sensors and enhanced biological senses come equipped with databases of scanned ``signatures'' that make it easier to identify whatever the user is sensing (in the case of bioware, these databases are stored and accessed via the character’s mesh inserts). For example, infrared sensors feature databases listing the heat signatures of different animals and items, making it easier to identify such things. In relevant situations, apply a +20 modifier for identifying targets sensed this way.

\subsubsection{Active vs. passive}

An active scanner must actually emit its particular frequency and then measure the reflections; this means a similar sensor can detect it and home in on the emitting source. For example, a character with enhanced vision can literally see the terahertz radiation emitted by someone using an active terahertz sensor, much like someone with normal vision can see the light emitted by a flashlight.

A passive scanner simply scans frequencies that occur naturally - there is nothing to give the sensor away.

\subsubsection{Electromagnetic spectrum}

For \textit{Eclipse Phase} rules purposes, the EM spectrum is broken down by wavelength and frequency into these categories: radio, microwave, terahertz, infrared, visible light, ultraviolet, X-rays, and gamma rays.

\textbf{Radar (Radio/Microwave): }Radar sensors work by actively emitting radio waves and microwaves and measuring them as they bounce off the target. Radar works best when detecting metallic objects, and is less effective (-20 modifier) against biomorphs and small items. Resolution is not high, however, so it can see shapes but not colors or fine details. It can be used to detect both speed and movement, can ``see'' through walls (up to a cumulative Armor + Durability of 100), and can detect cybernetic implants or concealed items. At close ranges (1-2 meters), it can detect pulse rate and respiration by measuring the motion of the chest cavity.

\textbf{Terahertz:} Terahertz sensors emit t-rays, measure the reflections, and compare them to a database of terahertz signatures that different items/materials have. The resolution is higher than radar, but with slightly less detail than normal vision. Similar to radar, terahertz sensors can see through walls and other materials, but to a lesser extent (up to a cumulative Armor + Durability of 50). T-rays occur naturally, but terahertz sensors normally require an emitter as they are absorbed by atmosphere (as well as water and metal). In space, however, an emitter would not be required. Likewise, passive terahertz scans within atmosphere have an effective range of 25 meters. T-rays do not penetrate skin, so are ineffective for locating implants.

\textbf{Infrared:} Near-infrared wavelengths are used for night vision, providing resolution and detail equivalent to regular vision under low-light conditions. Mid-long infrared is excellent for detecting heat sources (unobstructed by fog or smoke) and temperature differences (as small as 0.1 degree C), and such thermal imaging will sense the dissipating heat traces left by warm sources on colder ones, allowing the user to see where someone was sitting, trace fading heat footprints, or see what buttons were pressed if they are quick enough. Infrared also detects the blood flow in a biomorph’s face, which can be useful in judging emotional states (+20 modifier to Kinesics Tests), and can spot subsurface implants. Some normally white surfaces are reflective (mirrored) in infrared, potentially allowing an infrared viewer to see around corners or behind themselves. On the other hand, some glass is opaque to infrared light. Infrared is also useful for determining chemical composition (enabling Chemistry Tests by sight alone). Infrared sensory input is passive.

\textbf{Lidar (Visible Light):} Similar to radar, but with much higher resolution, lidar actively bounces light from the infrared through ultraviolet spectrum off a target and measures the backscatter, fluorescence, and other properties. Lidar is very useful for detecting atmospheric chemical properties and weather. Like radar, it can be used to measure a target’s range and speed, or develop a three-dimensional image. One clever use of lidar is to precisely ``map'' the position of everything in a room (taking several turns of scanning) and then check that positioning later to see if anything has been moved.

\textbf{Ultraviolet:} Some objects are fluorescent in ultraviolet light, including some animals, flowers, insects, urine, and minerals (which show up much better in ultraviolet than regular light). Some plants and animals have patterns that can only be seen in ultraviolet. Likewise, chemical dyes that only show up under ultraviolet, or that make certain substances (like blood) fluoresce under ultraviolet light, have various security purposes. Some glass is opaque at ultraviolet wavelengths.

\textbf{X-Ray/Gamma-Ray:} Backscatter imaging systems using X- and gamma-ray frequencies produce high-resolution three-dimensional images and are very useful for detecting concealed weapons and implants. Such imagers are very good at penetrating walls and metal (up to a cumulative Armor + Durability of 200, at least at levels safe to transhumans). These sensors can, of course, also detect the presence of harmful radiation.

\subsubsection{Soundwaves}

The transmission of vibrations through a medium, sound is broken down into infrasound (frequencies below standard human hearing), normal acoustic range, and ultrasound (frequencies above standard human hearing). Soundwaves do not propagate in vacuum.

\textbf{Ultrasound:} Ultrasound sonar operates much like radar, bouncing sound waves off a target and measuring the returning echoes. Ultrasound imaging is similarly low-resolution, showing shapes and movement but no colors and few details unless measured closely (1-2 meters). Ultrasound is good for identifying how dense a material is, however, can detect denser materials hidden beneath less dense ones. Many medical devices utilize ultrasound, and ultrasound sensors can also detect gas leaks, frictional motor noises, and similar mechanical emissions. Ultrasound sensors are typically unaffected by noise clutter from standard acoustic frequencies.

\textbf{Infrasound:} Infrasound travels much further than regular sound frequencies (hundreds of kilometers). Mechanical machinery, seismic disturbances, tornados, explosions, waterfalls, and certain weather phenomena create infrasound waves. Large animals such as elephants and whales use infrasound to communicate via the ground over large distances, though infrasound data transfer is too slow for complex communications.

\subsubsection{Combined sensor systems}

When used in combination, these sensor technologies can be potent. For example, the use lidar, thermal imaging, and radar can provide a threedimensional map of a building and everyone and everything inside.

\subsection{Nanoware}
\label{sec:nanoware}

\subsection{Cosmetic mods}
\label{sec:cosmetic-mods}

\subsection{Robotic enhancements}
\label{sec:robotic-enhancements}

\subsubsection{Armor}

\subsubsection{Mobility systems}

\subsubsection{Physical modifications}

\subsubsection{Sensors}

\subsection{Armor}
\label{sec:armor}

\subsection{Armor mods}
\label{sec:armor-mods}

\subsection{Communications}
\label{sec:communications}

\subsection{Neutrino communicators}
\label{sec:neutrino-communicators}

\subsection{Quantum farcasters}
\label{sec:quantum-farcasters}

\subsection{Quantum entanglement communication}
\label{sec:quantum-entanglement-communication}

\subsection{Covert and espionage technologies}
\label{sec:covert-espionage-tech}

\subsection{Bugs and surveilance}
\label{sec:bugs-surveilance}

\section{Drugs, chemicals and toxins}
\label{sec:drugs-chemicals-toxins}

\subsection{Substance rules}
\label{sec:substance-rules}

\subsubsection{Classification of substances}

\subsubsection{Application methods}

\subsubsection{Drug effects}

\subsubsection{Addiction and substance abuse}

\subsection{Drugs}
\label{sec:drugs}

\subsubsection{Cognitive drugs}

\subsubsection{Combat drugs}

\subsubsection{Health drugs}

\subsubsection{Recreational drugs}

\subsubsection{Social drugs}

\subsection{Nanodrugs}
\label{sec:nanodrugs}

\subsubsection{Petals}

\subsubsection{Sample petals}

\subsubsection{Other nanodrugs}


\subsection{Sample petals}
\label{sec:sample-petals}

\subsection{Narcoalgorithms}
\label{sec:narcoalgorithms}

\subsection{Chemicals}
\label{sec:chemicals}

\subsection{Toxins}
\label{sec:toxins}

\subsection{Nanotoxins}
\label{sec:nanotoxins}

\subsection{Pathogens}
\label{sec:pathogens}

\subsection{Psi drugs}
\label{sec:psi-drugs}

\section{Everyday technology}
\label{sec:everyday-tech}

\section{Nanotechnology}
\label{sec:nanotech}

\subsection{Basic nanotechnology}
\label{sec:basic-nanotech}

\subsubsection{Healing vats}

\subsubsection{Nanodetectors}

\subsubsection{Nanofabricators}

\subsection{Advanced nanotechnology}
\label{sec:advanced-nanotech}

\subsubsection{Ego bridges}

\subsubsection{Nanoswarms and microswarms}

\subsection{Pets}
\label{sec:pets}

\subsection{Scavenger tech}
\label{sec:scavenger-tech}

\subsection{Services}
\label{sec:services}

\subsection{Software}
\label{sec:software}

\subsection{Survival gear}
\label{sec:survival-gear}

\section{Weapons}
\label{sec:weapons}

\subsection{Malee weapons}
\label{sec:melee-weapons}

\subsection{Kinetic weapons}
\label{sec:kinetic-weapons}

\subsection{Brand name weapons and combined arms}
\label{sec:brand-weapons-combined}

\subsection{Beam weapons}
\label{sec:beam-weapons}

\subsection{Seekers}
\label{sec:seekers}

\subsection{Spray weapons}
\label{sec:spray-weapons}

\subsection{Grenades and seekers}
\label{sec:grenades-seekers}

\subsection{Exotic ranged weapons}
\label{sec:exotic-ranged-weapons}

\subsection{Weapon accessories}
\label{sec:weapon-accessories}

\section{Robots and vehicles}
\label{sec:robots-vehicles}

\subsection{Aircraft}
\label{sec:aircraft}

\subsection{Exoskeletons}
\label{sec:exoskeletons}

\subsection{Groundcraft}
\label{sec:groundcraft}

\subsection{Personal vehicles}
\label{sec:personal-vehicles}

\subsection{Robots}
\label{sec:robots}

\subsection{Spacecraft}
\label{sec:spacecraft}

%%% Local Variables: %%% mode: latex %%% TeX-master: "ep" %%% End: 
