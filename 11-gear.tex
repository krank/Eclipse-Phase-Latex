%%% EDITION: 2nd Printing

\chapter{Gear}
\label{cha:gear}

The accelerated technological levels of \emph{Eclipse Phase} enable a number of devices for personal enhancement, survival, and other uses.


\section{Equipment rules}
\label{sec:equipment-rules}

The following rules apply to all technological items in \emph{Eclipse Phase}.


\subsection{Acquiring gear}
\label{sec:acquiring-gear}

During character creation, players purchase gear for their characters using the credits they have during the character creation process. Once play begins, however, characters must obtain any equipment they need the usual way: by buying, borrowing, making, or stealing it.

In the inner system, hypercorp, and Jovian Republic settlements --- and other places where capitalism still reigns --- gear acquisition is simply a matter of finding a seller and buying it. Each item has a listed cost, from Trivial to Expensive, as noted on the Gear Costs table. Due to local availability of resources, supply and demand, and legalities, these listed costs are meant to be approximations. When no other factors apply, the listed Average Cost for that category can be used. Otherwise the gamemaster should modify the item’s worth as they see fit, according to local economic factors, while still keeping it within that cost category range. The Cost Modifiers table lists out some suggested changes to an item’s cost, but these are simply recommendations, and can be ignored or followed as the gamemaster deems fit. The exact local conditions are largely up to the gamemaster to determine, as best fits their game.

In some circumstances, characters may attempt to haggle over gear prices. This is best handled as roleplaying, but the gamemaster may also call for an Opposed Persuasion Test (or possibly an Intimidation Test). The character who wins may increase or reduce the price by 10\% per 10 points of MoS.

In the outer system, anarchist, Titanian, scum, and other habitats that use the reputation economy, characters must rely on their rep scores to acquire the goods and services they need. The mechanics for this are covered under \emph{Reputation and Social Networks}, p. 285.

Characters are of course free to get their hands on equipment by any other means they devise --- con schemes, borrowing from friends, and outright theft, with all of the appropriate tests and consequences. In some cases, acquiring gear may be an adventure unto itself.

\begin{table}
\begin{tabular}{|l|r|r|}
\hline
\multicolumn{3}{|c|}{\textbf{Gear costs}}			\\
\hline
\textbf{Category}	& \textbf{Range (in credits)}	& \textbf{Average (in credits)} \\
\hline
Trivial			& 1-99					& 50 \\
\hline
Low				& 100-499					& 250 \\
\hline
Moderate			& 500-1499				& 1000 \\
\hline
High				& 1500-9999				& 5000 \\
\hline
Expensive			& 10000+					& 20000 \\
\hline
\end{tabular}
\label{tab:gear-costs}
\end{table}

\begin{table}
\begin{tabular}{|l|r|}
\hline
\multicolumn{2}{|c|}{\textbf{Gear cost modifiers}}			\\
\hline
\textbf{Economic factor}	& \textbf{Suggested cost modifier} \\
\hline
Item Stolen			& -50\% \\
\hline
Item Used				& -25\% \\
\hline
Item Restricted		& +25\% \\
\hline
Item Illegal			& +50\% \\
\hline
Item Scarce			& +25\% \\
\hline
Item Extremely Rare		& +50\% \\
\hline
Item Common			& -25\% \\
\hline
\end{tabular}
\label{tab:gear-cost-modifiers}
\end{table}

\subsubsection{Fabricating gear}

Thanks to nanofabrication technology, characters may also create their own equipment using cornucopia machines and similar nanofab devices (p. 327). The character must have the appropriate blueprints to do so, whether they come with the fabber, are bought legitimately or on the black market, acquired with rep, or found online. Characters may also code their own blueprint desires, using the Programming: Nanofabrication skill.


\subsection{Gear modifiers}
\label{sec:gear-modifiers}

In the technological future, gear is a necessity. In many cases, use of equipment provides no bonuses, it simply allows a character to perform a task they would otherwise be unable to do. For example, it is impossible to pick a mechanical lock without lockpick or some sort of tool.

In other cases, however, gear provides a bonus to the task at hand. Climbing a wall may be possible without tools, but if you happen to have gecko gloves or other climbing gear, it’s going to be a lot easier. The specific modifier applied is usually noted in the gear item’s description, typically ranging from +10 to +30.

\subsubsection{Gear quality}

In both of the situations above, it is possible to have items that are of either exceptional or inferior quality, with corresponding positive or negative modifiers. The gear may be well-crafted, state-of-the-art, cutting-edge experimental, or simply top-of-the-line, applying an additional +10 to +30. Or it may be outdated, shoddy, or in disrepair, inflicting a -10 to -30 modifier (in some cases canceling out the basic gear bonus).

\subsubsection{Gear sizes}

On occasion, you’ll need to know how small or large a certain piece of equipment is. Though this is largely something the gamemaster can wing on the fly using common sense, we’ve listed sizes for many gear items that are unusual or so futuristic that the average player may not have a feel for what dimensions the tech likely is. These size categories are listed on the Gear Sizes table (p. 297). These sizes should be considered approximations, as depending on the manufacturer and process, some items may be smaller or larger than similar items. It is also important to keep in mind that as technology advances, the size and components of various equipment items shrink, so when in doubt, go with smaller.


\begin{table}
\begin{tabularx}{\textwidth}{|l|X|}
\hline
\multicolumn{2}{|c|}{\textbf{Gear sizes}}			\\
\hline
\textbf{Size category}	& \textbf{General dimensions and notes} \\ Nano					& So small that the item cannot be seen without the aid of a microscope or nanoscopic vision (p. 311), and may not be manipulated without fractal digits (p. 311) or similar tools. \\
\hline
Micro				& Anything ranging from the size of a barely visible small dot to an average insect. \\
\hline
Mini					& Mini items may be concealed within someone’s palm or small pockets. \\
\hline
Small				& Small items may be held in one hand and concealed in normal pockets.\\
\hline
Medium				& Medium size items are cumbersome to hold with one hand, ranging from the size of a 2-liter bottle to the size of a medium dog. They do not fit in pockets, but they may be concealed by larger coverings. \\
\hline
Large				& Roughly human-sized. \\
\hline
Huge					& Vehicles and other more massive objects. \\
\hline
\end{tabularx}
\label{tab:gear-sizes}
\end{table}


\subsubsection{Mass and encumbrance}

A character who is carrying too much gear should be slowed down, suffering negative modifiers both to their movement rates and their skill tests. Rather than micromanaging the weights of individual pieces of equipment, however, this matter is largely left to the gamemaster’s discretion, using common sense. If a character loads up beyond reason, apply modifiers as seem appropriate. The gamemaster should, however, keep in mind that many of the manufacturing materials used in \emph{Eclipse Phase} allow for items that are much lighter than current standards without any loss of durability or function (see \emph{Future Materials}, p. 298). Likewise, characters in low or microgravity environments can carry much larger loads.

\subsubsection{Concealing gear}

Characters may attempt to conceal items on their person, hoping at least to hide them from casual notice if not an intensive search. To determine how effectively the character conceals the equipment, make a Palming Test and note the MoS (the gamemaster may wish to roll this secretly). Whenever another character has a chance to notice the concealed item, they must succeed in a Perception Test and achieve a higher MoS than was scored on the Palming Test. The gamemaster should apply modifiers to both tests as appropriate. For example, concealing a large item like a sword would be difficult (-30), whereas wearing concealing clothing like a longcoat or multi-pocketed jumpsuit would help (+20). Likewise, a character who is not actively looking is less likely to notice the hidden gear (-30), whereas someone who conducts a physical search (+30) or who has enhanced vision to pierce protective layers will fare better.


\subsection{Design and fashion}
\label{sec:design-fashion}

Many objects in \emph{Eclipse Phase} closely resemble their early 21st century equivalents --- a bottle of soda is still a transparent container holding a brightly colored liquid, clothing is obviously something you wear, and a knife still consists of a blade and a handle. The materials, processes, and mindsets that go into making them, however, are quite different. To start, very few items look have a uniform, mass-produced look, even if they were. The procedures of minifacturing and nanofabrication allow every individual item to be manufactured with a unique (or at least different) look. In areas with anarchist/reputation economies, in fact, where personal possessions have very little intrinsic value, expression and creativity are favored and so many items are artistically personalized (and actual hand-crafted items are rare and prized). Likewise, almost all equipment is designed with ergonomics and ease-of-use prioritized, so gear with soft curves, pleasing colors, and form-fitting shapes are common. Many items of personal technology, such as flashlights or small tools, are made in the form of ovoids that fit comfortably in the user’s hand or in similar forms that can be easily worn or attached to clothing. To someone from the 20th century, many common devices look like oddly colored rocks or decorative pieces of plastic or ceramic (in fact, many such items are referred to as ``blobjects'' by older transhumans).

The materials used to create everyday items are also advanced, ranging from aerogel and graphene to smart materials (p. 298) and exotic metamaterials with unusual physical properties. In practice, this means that most items are light, durable (with both tensile strength and/or flexibility, as needed), waterproof, dirt-repellent, and self-cleaning. Most gear is also designed with zero-G or microgravity functionality in mind, and can easily be clipped, tethered, or stuck to a surface with grip pads.

Almost all gear available in \emph{Eclipse Phase} is also available in forms that are wearable/usable by uplifted animals and non-humanoid morphs, such as novacrabs, slitheroids, and so on. Even if such customized gear is not immediately available, it is usually not difficult to nanofabricate. Smart materials (p. 298) also make interoperability between different morphs easy.

\subsubsection{Interface}

It is not uncommon for everyday devices to have no visible controls as they are designed to be operated via radio broadcasts from the user’s ecto or mesh inserts. Any items crafted for use in emergency, combat, survival, or exploration situations, however, will feature basic physical controls, just in case. Physical interfaces are typically controlled by touch pads that are nothing more than colored spots on the device’s surface, though some may also project a holographic interface display. Most equipment of this sort can can also be voice-activated and controlled.

Almost all devices are loaded with a complete set of help files and tutorials. Most electronics are also mesh-capable and equipped with specialized AIs (see \emph{Meshed Gear}, next page).

\subsubsection{Smart materials}

Many common items of technology are made from so-called smart materials. These devices contain --- or sometimes consist entirely of --- many small nanomachines that can both move and reshape themselves to alter the object’s shape, color, and texture. For example, smart clothing can transform from a suit of specialized cold weather clothing suitable for the Martian poles in winter to a fashionable suit in the latest style due to hundreds of thousands of tiny nanomachines in the clothing that shift and move to reshape the garment. Similarly, a tool made of smart materials can switch from a powered screwdriver to a wrench or a hammer, as the nanomachines move around and completely reshape the tool. Smart materials all contain specialized advanced nanomachine generators (p. 328) that keep them in perfect repair as long as they are regularly recharged.


\subsection{Future materials}
\label{sec:future-materials}

Many materials are available and commonly used in \emph{Eclipse Phase} that are rare, theorized, or unheardof today. The following entries note some of the more interesting.

\subsubsection{Aerogel}

Low-density, solid-state ``Frozen smoke'' is made by carefully foaming various materials, typically glasses or ceramics, to an ultra-low density state. Aerogel is semi-transparent and light-weight, feels like styrofoam, but acts as an incredible insulator against heat and cold. It is commonly used in habitats.

\subsubsection{Diamond}

Artificial diamond is lightweight and super-strong, has an extremely high melting point, and has nearperfect thermal conductivity. This makes it an ideal substance for hardening coated surfaces (armor) and creating super-tough diamond machinery.

\subsubsection{Fullerenes/Fullerites}

Fullerenes are molecular carbon structures (known as buckyballs, carbon nanotubes, and graphene) that are extremely strong (vastly stronger by weight than steel), heat-resistant, and can be either insulative or superconductive. This makes them useful in equipment as diverse as armor, electronics, sensor systems, or the cables of space elevators.

\subsubsection{Metallic foam}

Metal foam is created by adding foaming agents to liquid metals, resulting in extremely lightweight metallic structures --- light enough to float on water. Ideal for habitat construction and floating cities.

\subsubsection{Metallic glass}

Metallic glass are metals highly alloyed to possess a disordered (rather than crystalline) atomic structure with unique combinations of stiffness and strength, making it a good wear surface and alternative to ceramics in armor. It is also useful for its unusual (for a metal) electrical resistance properties.

\subsubsection{Metamaterials}

Metamaterials have unusual physical properties (usually electromagnetic) due to their structure, such as having a negative refractive index. Metamaterials are used to create invisibility cloaks (p. 316), superlenses, phased array optics, and impressive 2-D holograms.

\subsubsection{Refactory metals}

These metallic alloys have extremely high melting points, making them ideal for extremely hot engine systems, atmospheric entry vehicles, and hypersonic craft.

\subsubsection{Transparent alumina}

In transparent form, this ceramic is often known as sapphire. Transparent alumina is harder than steel and zero-g casting techniques allow for intriguing transparent construction designs, so long as its poor tensile strength is respected.


\section{Meshed gear}
\label{sec:meshed-gear}

Almost all technology in \emph{Eclipse Phase} is designed to be operated via radio signals from the user’s basic implant, although models usable by characters without basic implants are also available. In addition all devices contain a nearly microscopic computer and radio link (known as a ``voice'') that allows the user to easily locate the object and that reports on the condition of the object or device, how to properly use and care for it, as well as telling the user when it needs to be repaired and how. Most are discrete and highly useful, but cheaply made goods sometimes have overly annoying voices.

This means that almost all devices can be accessed via the mesh or directly if within radio range. This makes them vulnerable to hacking and intrusion attempts (p. 254) as well as radio jamming (p. 262). Many devices are, however, publicly accessible (see \emph{Spimes}, p. 238). Meshed gear may also be tracked through the mesh (p. 251). For privacy and security, these devices are often slaved to other systems (see \emph{Slaving Devices}, p. 248); devices worn/carried by characters are usually made part of the personal area network and slaved to the character’s mesh inserts/ ecto. For more info on meshed devices, see the \emph{Mesh chapter}, p. 234.

Many devices come equipped with AIs, who are equipped with skillsofts that enable them to operate the device on their own, as according to voiced instructions or commands issued through the net. AIs are described on p. 264 and p. 331.


\subsection{Radio and sensor ranges}
\label{sec:radio-sensor-ranges}

In \emph{Eclipse Phase}, almost all devices are equipped with small radios so that they may be meshed. Likewise, many pieces of gear are equipped with sensors such as cameras, microphones, or other detectors. The Radio and Sensor Ranges table notes what range these devices operate at.

\begin{table}
\begin{tabularx}{\textwidth}{|l|l|l|X|}
\hline
\multicolumn{4}{|c|}{\textbf{Radio and sensor ranges}}			\\
\hline
\textbf{Size category}	& \textbf{Urban range}	& \textbf{Urban range}	& \textbf{Examples} \\
\hline
Nano 				& 20 meters 			& 100 meters			& Smart Dust, Nanobot/Microbot Swarms \\ Micro				& 50 meters			& 500 meters			& Microbugs \\ Mini					& 1 kilometer			& 20 kilometers		& Mesh Inserts \\ Small				& 5 kilometers			& 50 kilometers		& Ectos, Miniature Radio Farcasters, Portable Sensors \\ Medium				& 25 kilometers		& 250 kilometers		& Radio Boosters, Vehicle Sensors \\ Large				& 500 kilometers		& 5000 kilometers		& \\
\hline
\end{tabularx}
\label{tab:radio-sensor-ranges}
\end{table}


\subsection{Power}
\label{sec:power}

All of the powered devices in \emph{Eclipse Phase} require electricity to function. With rare exceptions, most of them rely on either solar cells or powerful batteries. These batteries are high-density, room-temperature superconductors with 25 times the capacity of the best batteries in common use in the early 21st century. Such batteries may also be constructed so that they are flexible, printed on devices, or woven into fabric. They are good for 100-500 hours of use, and will alert the user when they start running low. More powerful radio-isotope nuclear batteries are also available, heavily shielded so they do not emit radiation and good for 3 years or more of use.

In short, power should rarely be an issue in \emph{Eclipse Phase} games, unless it happens to fit the plot. Power failure could also result from a critical failure roll.


\section{Personal augmentation}
\label{sec:personal-augmentation}

Almost all citizens of the solar system, whether human, AI, or uplifted animal, use various forms of biological, cybernetic, or nanotechnological augmentation. The following is a list of the most common types.

Unless otherwise noted, any bonuses from personal augmentations are both compatible and cumulative with bonuses from other enhancements.


\subsection{Standard augmentations}
\label{sec:std-augmentations}

Most morphs produced in the solar system include the following augmentations.

\subsubsection{Basic biomods}

Almost universal in biomorphs, many habitats will not allow individuals to visit/immigrate if their biomorph does not possess these biomods in order to preserve public health. Basic biomods consists of a series of genetic tweaks, tailored virii, and bacteria that speed healing, greatly increase disease resistance, and impede aging. A morph with basic biomods heals twice as fast as an early 21st century human, gradually regrows lost body parts, is immune to all normal diseases (from cancer to the flu), and is largely immune to aging. In addition, the morph requires no more than 3-4 hours of sleep per night, is immune to ill-effects from longterm exposure to low or zero gravity, and does not naturally suffer from biological problems like depression, shock reactions after being injured, or allergies. \textbf{[Moderate, but included for free in most biomorphs]}

\subsubsection{Basic mesh inserts}

Mesh inserts are ubiquitous among modern morphs. This network of cybernetic brain implants is essential equipment for anyone who wants to stay connected and make full use of the wireless mesh. The interconnected components of this system include:

\begin{itemize}
\item \textbf{Cranial computer:} This computer serves as the hub for the character’s personal area network and is home to their muse (p. 264). It has all of the functions of a smartphone and PDA, acting as a media player, meshbrowser, alarm clock/calendar, positioning and map system, address book, advanced calculator, file storage system, search engine, social networking client, messaging program, and note pad. It manages the user’s augmented reality input and can run any software the character desires (see \emph{Software}, p. 331). It also processes XP data, allowing the user to experience other people’s recorded memories, and also allowing the user to share their own XP sensory input with others in real-time. Facial/image recognition and encryption software (p. 331) are included by default.
\item \textbf{Radio tranciever:} This transceiver connects the user to the mesh and other characters/devices within range. It has an effective range of 20 kilometers in deep space or other locations far from radio interference and 1 kilometer in crowded habitats.
\item \textbf{Medical sensors:} This array of implants monitors the user’s medical status, including heart rate, respiration, blood pressure, temperature, neural activity, and much more. A sophisticated medical diagnostic system interprets the data and warns the user of any concerns or dangers.
\end{itemize}

Using any of these functions is as easy as thinking. \textbf{[Moderate, but included for free in most morphs]}

\subsubsection{Cortical stack} A cortical stack is a tiny cyberware data storage unit protected within a synthdiamond case the size of a grape, implanted at the base of the skull where the brain stem and spinal cord connect. It contains a digital backup of that character’s ego. Part nanoware, the implant maintains a network of nanobots that monitor synaptic connections and brain architecture, noting any changes and updating the ego backup in real time, right up to the moment of death. If the character dies, the cortical stack can be recovered and they may be restored from the backup (see Resleeving, p. 271). Cortical stacks do not have external or wireless access (for security), they must be surgically removed (see Retrieving a Cortical Stack, p. 268). Cortical stacks are extremely durable, requiring special effort to damage or destroy. They are commonly recovered from bodies that have otherwise been pulped or mangled. Cortical stacks are intentionally isolated from mesh inserts and other implants, as a security measure to prevent hacking or external tampering. \textbf{[Moderate, but included for free with most morphs]}

\subsubsection{Cyberbrain}

Cybernetic brains are where the ego (or controlling AI) resides in synthmorphs and pods. Modeled on biological brains, cyberbrains have a holistic architecture and serve as the command node and central processing point for sensory input and decision-making. Only one ego or AI may ``inhabit'' a cyberbrain at a time; to accommodate extras, mesh inserts (p. 300) or a ghostrider module (p. 307) must be used. Since cyberbrains store memories digitally, they have the equivalent of mnemonic augmentation (p. 307). They also have a built-in puppet sock (p. 307) may be remote-controlled, though this option may be removed by those who value their security. Cyberbrains are vulnerable to brainhacking (p. 261) and other forms of electronic infiltration/attack. Cyberbrains come equipped with two or more pairs of external access jacks (p. 306), usually located at the base of the skull, which allow for direct wired connections. \textbf{[Moderate, but included for free in all synthetic morphs and pods]}

\subsection{Bioware}
\label{sec:bioware}

Bioware augmentations can be acquired either as a genemod when the morph is designed and grown or as a later modification to an existing morph, either by using nanomachines to modify the morph’s tissue or by externally growing the organ and implanting it. Bioware may be used to enhance biomorphs (including pods and uplifts), but not synthmorphs. Bioware may be used to enhance biomorphs (including pods and uplifts), but not synthmorphs (see Synthmorphs and Bioware, p. 306).

\subsubsection{Enhanced senses}

The following are a list of the most common enhanced senses. Each is also available as a cybernetic implant, but bioware is much more common.

\textbf{Direction Sense:} The character has an innate sense of direction and distance using advanced inertial navigation. The character can arbitrarily define any point as ``north'' and keep track of which direction that is, as well as knowing approximately how far they have come. Characters with this augmentation can always retrace any route they have taken, only experiencing difficulty with three-dimensional routes lacking navigational markers (such as deep space or undersea; apply a -30 modifier). Since positioning inside habitats by anyone with basic mesh inserts is an automatic affair, only characters venturing to remote locations require this augmentation. \textbf{[Low]}

\textbf{Echolocation:} The character possesses sonar similar to that of a bat or dolphin. The character bounces brief ultrasonic pulses off their surroundings and uses them to form an image of these surroundings through the pattern of reflections of these pulses received by the character’s ears. For more details, see Using Enhanced Senses, p. 302. This augmentation works in both air and water and has a range of 20 meters in air and 100 meters in water. \textbf{[Low]}

\textbf{Enhanced Hearing:} The morph’s ears are enhanced to hear both higher and lower frequency sounds --- the range of sounds they can hear is twice that of normal human ears (see Using Enhanced Senses, p. 302). In addition, their hearing is considerably more sensitive, allowing them to hear sounds as if they were five times closer than they are. A character with this augmentation can easily overhear even a softly spoken conversation at another table in a small restaurant. This augmentation provides a +20 modifier to all Perception Tests involving hearing. \textbf{[Low]}

\textbf{Enhanced Smell:} The morph’s sense of smell is equal to that of a bloodhound. The user can identify both chemicals and individuals by smell, and can track people and chemically reactive objects by smell as long as the trail was made within the last several hours and has not been obscured. The character can also gain a general sense of the emotions and health of any character within 5 meters (+20 to Perception or Kinesics Tests to do so). \textbf{[Low]}

\textbf{Enhanced Vision:} The morph’s eyes have tetrachromatic vision capable of exceptional color differentiation. These eyes can also see the electromagnetic spectrum from terahertz wave frequencies to gamma rays, enabling them to see a total of several dozen colors, instead of the seven ordinary human eyes can perceive. In addition, these eyes have a variable focus equivalent to 5 power magnifiers or binoculars. This augmentation provides a +20 modifier to all Perception Tests involving vision. For further applications, see Using Enhances Senses, p. 302. \textbf{[Low]}

\subsubsection{Mental augmentations}

Mental augmentations are extremely common.

\textbf{Eidetic Memory:} The character can remember everything that ever happened to them, in detail, with no long term memory loss. For example, they can recite a page they read in a book a month ago, recall a string of 200 random characters they viewed a year ago, or even tell you what they had for breakfast on a particular date a decade ago. However, they can only remember things they paid attention to. The character will not remember the contents of a note on someone’s desk if they merely glanced at it; they must specifically have read it. No effort is required to use this augmentation, the character merely needs to attempt to remember a specific fact. \textbf{[Low]}

\textbf{Hyper Linguist:} The morph’s brain maintains the linguistic flexibility of a small child, allowing the character to learn languages with great ease. This functions as the Hyper Linguist trait, p. 146. \textbf{[Low]}

\textbf{Math Boost:} This implants functions as the Math Wiz trait, p. 146. \textbf{[Low]}

\textbf{Multiple Personalities:} The character’s brain is intentionally partitioned to accommodate an extra personality. This multiplicity is not viewed as a disorder, but as a cognitive tool to help people deal with their hypercomplex environments. This extra personality can be an NPC run by the gamemaster, a separate character (in ego form only) made by the player, or the downloaded fork of another character. For all intents and purposes, the extra personality is treated as a separate ego (i.e., it may fork separately), except that both personalities are backed up in the same cortical stack and if downloaded they must be placed in separate morphs or in another morph with this implant.

Only one ego may be in control of the morph at a time. The other resides in the background, still active, but not on a surface level. Each ego is completely aware of what the other is doing, thinking, etc. If for some reason the subsumed personality wants to come to the fore, but the other personality won’t relinquish control, make an Opposed WIL x 3 Test. Each ego has its own Lucidity and Trauma Threshold, and they track stress and trauma separately. Any psi attacks or social/ mental influences only affect the personality at the fore. Having an extra ego in your head, working in the background, is helpful for multitasking. The character receives an extra Complex Action each turn that may only be used for mental or mesh actions. \textbf{[High]}

\subsubsection{Physical augmentations}

Most physical bioware augmentations are derived from the capabilities of animals.

\textbf{Adrenal Boost:} This adrenal gland enhancement supercharges the character’s adrenal response to situations that invoke stress, pain, or strong emotions (fear, anger, lust, hate). When activated, the concentrated burst of norepinephrine accelerates heart rate and blood flow and burns carbohydrates. In game terms, this allows the character to ignore the -10 modifier from 1 wound and temporarily increases REF by +10 (also boosting REF-linked skills and Initiative). These modifiers apply until the character has calmed down (if the character also has endocrine control, p. 304, then adrenal boosts can be activated and deactivated at will, and the negated wounds are cumulative). \textbf{[High]}

\textbf{Bioweave Armor (Light):} Bioweave armor involves lacing the morph’s skin with artificial spider silk biological fibers. This provides an Armor rating of 2/3 without changing the appearance, texture, or sensitivity of the morph’s skin. This armor is cumulative with worn armor. \textbf{[Low]}

\textbf{Bioweave Armor (Heavy):} Heavy bioweave armor involves lacing the morph’s skin with a denser and thicker network of the same fibers. The morph’s skin becomes thicker and somewhat less flexible except at the joints. The morph’s skin also has an unusually smooth look, and a distinctively smooth and tough-feeling texture. This provides an Armor rating of 3/4 without decreasing the morph’s mobility. The character’s sense of touch, however, is significantly reduced (-20 modifier) except on their hands, feet, and face. This armor is cumulative with worn armor. \textbf{[Moderate]}

\textbf{Carapace Armor:} Carapace armor combines bioweave armor with hard but flexible plates of a chitin-ceramic hybrid material modeled on the microscopic structure and texture of arthropod exoskeletons. This armor is obvious and has a somewhat crocodilian or insectoid appearance (character’s choice). The morph is completely hairless as well. This provides an Armor rating of 11/11. This armor is not cumulative with worn armor. \textbf{[Moderate]}

\textbf{Chameleon Skin:} The morph’s skin is augmented with complex chromatophores so that it changes color like the skin of a chameleon or an octopus. The morph can match the appearance of almost any color and most patterns. This provides a +20 modifier to Infiltration Tests to avoid being seen or noticed, as long as the character is stationary or not moving faster than a slow walk. The character must be nude or wearing smart clothing (p. 325) of the same color/pattern. If incompletely camouflaged, or if moving faster, reduce the modifier to +10. In addition to blending in, the character can also consciously change the color and pattern of their skin to deliberately stand out (+20 on Perception Tests to notice) or simply to produce attractive or interesting colors or patterns. \textbf{[Low]}

\textbf{Circadian Regulation:} The morph only requires 2 hours of sleep to maintain health and function at peak mental capacity. The character dreams constantly while asleep and can both fall asleep and wake up almost instantly. In addition, the character can easily and with no ill-effects shift to a 2-day cycle, where they are awake for 44 hours and sleep for 4. \textbf{[Moderate]}

\textbf{Claws:} The morph has retractable claws like those of a cat. These claws do not interfere with the character’s manual dexterity and are razor sharp. However, they are relatively small and only do 1d10 + 1 + (SOM $\div$ 10) damage, with an AP of -1. As a result, they are legal in almost all habitats and are considered tools as much as weapons. \textbf{[Low]}

\textbf{Clean Metabolism:} The morph’s symbiotic bacteria, gut flora, and glands have been genetically engineered to keep the morph ``clean.'' The morph also produces smart antibiotics that prevent the growth of any bacteria or yeasts in it or on its skin. As a result, the morph is completely immune to infections, dental cavities, and bad breath, its sweat has no scent, and the morph’s efficient digestion produces somewhat less solid waste and less odorous chemicals. \textbf{[Moderate]}

\textbf{Drug Glands:} The morph has specially-tailored glands designed to produce specific hormones or chemicals and release them in the body. The character has control over these glands and can release the chemicals at will. Each type of drug gland is considered a separate enhancement. For potential drugs and chemicals, see p. 317. \textbf{[One Cost Category Higher Than Drug Cost]}

\textbf{Eelware:} Derived from electric eel genetics, a character can have eelware implanted so that it connects to a network of bioconductors in the hands and feet (or other limbs), allowing the character to generate stunning shocks with a touch. Eelware inflicts shock damage (p. 204) exactly like a pair of shock gloves. Eelware can also be used to power implants and specially designed handheld devices by touch. \textbf{[Low]}

\textbf{Emotional Dampers:} This low-cost alternative to endocrine control (p. 304) allows the user to voluntarily damp their morph’s emotional responses and various non-verbal cues like pupil dilation, eye movement, or vocal tone. Using this augmentation allows the user to lie and conceal their emotions in such as way as oo fool the keenest observer; apply a +30 modifier to Deception and Impersonation Tests. This modification does not affect methods of detecting lies and emotions that involve reading the character’s neural state, including psi-gamma sleights. However, this augmentation damps out all emotional responses and so causes the character to be less persuasive in real- time personal interactions, imposing a -10 modifier to other Social skill tests like Persuasion. Characters can turn this augmentation on or off at will. \textbf{[Low]}

\textbf{Endocrine Control:} This augmentation modifies the morph’s endocrine system, giving the character fine control over their hormone output. This allows the character to completely control their appetite and emotions and to regulate pain. They receive a +30 modifier against the effects of hunger, fear, and any forms of emotional manipulation, such as the Drive Emotion sleight. This augmentation also allows character to lie with perfect conviction and to completely fool all methods of lie detection that do not rely on the target’s neural output; apply a +20 modifier to Deception Tests. It also allows the character to remain awake for 48 hours without penalty, but after this time the character begins experiencing normal fatigue. Finally, the ability to regulate pain reception allows the character to ignore the -10 modifier from 1 wound. \textbf{[High]}

\textbf{Enhanced Pheromones: }The morph’s biochemistry has been altered so that it produces enhanced pheromonal signals that subconsciously affect the behavior of other humans in the vicinity. These pheromones make the character more attractive and trustworthy to the target; apply a +10 modifier to appropriate Social skill tests, such as Persuasion. This augmentation only affects characters who can smell the pheromones, and it does not affect uplifts or xenomorphs. [Low] Enhanced Respiration: By boosting both lung efficiency and the blood’s oxygen-carrying capacity, the character can live comfortably in both high and low pressure environments, from 0.2 atmospheres to 5 atmospheres, with no dizziness or need for gradual decompression. In addition, the character can hold their breath for up to 30 minutes when performing minimal activity or for up to 10 minutes while performing highly strenuous activity. \textbf{[Low]}

\textbf{Gills:} The morph’s lung tissue has been adapted to function as gills, allowing the morph to breathe both air and water, as long as the water is not toxic or too stagnant. Characters with this augmentation breathe in water and then expel the water through slits just underneath their lowest pair of ribs that seal when the character is not underwater. \textbf{[Low]}

\textbf{Grip Pads:} The morph possesses specialized pads on its palms, lower arms, shins, and the bottoms of its feet. Designed to emulate the pads on gecko feet, characters can support themselves on a wall or ceiling by placing any two of these pads against any surface not made from a material specially designed to resist this augmentation. Characters can climb any surface and move easily across ceilings that can support their weight. Apply a +30 modifier to Climbing Tests. The pads must be free to touch the surface the character is climbing (no gloves). The nature of these pads is obvious to anyone looking at them, but they do not impair the character’s sense of touch or manual dexterity. If combined with the vacuum sealing augmentation, the character can even stick to surfaces in the vacuum of space. \textbf{[Low]}

\textbf{Hibernation:} The character can voluntarily reduce the morph’s metabolism to the point that the morph requires only 5\% of the normal amount of food, water, and air. The character appears to sink into a deep sleep, but can maintain a dim awareness of both touch and sound and so can be easily awakened. Entering or leaving this state requires 3 minutes where the character is relatively helpless. With sufficient air, characters can safely hibernate for up to 40 days without food or water. \textbf{[Low]}

\textbf{Muscle Augmentation:} The morph’s muscle mass has been enhanced and toned and myofibers strengthened. Apply a +5 modifier to SOM. \textbf{[High]}

\textbf{Neurachem:} This bioware modification enhances the character’s chemical synapses and juices their neurotransmitters, drastically speeding up neural connections. Neurachem can be mentally activated or triggered by charged emotions. Level 1 neurachem increases the character’s Speed stat by +1, with no side effect. Level 2 raises the Speed stat by +2, but each time it is used the character suffers a nervous system fatigue hangover for 1 hour after the boost wears off (apply a -20 modifier to all actions). \textbf{[High (Level 1), Expensive (Level 2)]}

\textbf{Poison Gland:} Similar to the drug gland, this morph has special glands that produce poisons, like the venom glands of a snake. The morph has poison glands in its fingers and mouth, so that it can deliver either poison by scratching someone with a fingernail, biting them hard enough to draw blood, or even by sharing a beverage with someone or spitting into their drink. The morph is immune to the poisons it produces. These glands may not produce nanotoxins. \textbf{[Low]}

\textbf{Prehensile Feet:} The morph’s feet and leg joints are altered so that its toes are longer and more dexterous and the big toe is transformed into an opposable thumb. Physically, the morph’s feet resemble a longer narrower hand or a human foot with finger (and thumb)-like toes. The character can walk normally but must wear specially designed shoes. However, this morph runs somewhat slower than a morph with unmodified feet (-1 meter per Action Turn). In addition, the morph’s hips are slightly modified to allow greater mobility. In a properly constructed chair, or when floating in zero-G, the character can use both their hands and their feet to manipulate the same object. Most morphs used by characters who live in zero-G possess this augmentation. \textbf{[Low]}

\textbf{Prehensile Tail:} A long (1.5 meters) prehensile tail is added to the morph’s backside, extending out from the tailbone. This tail is prehensile and may be used to grab, hold, and even manipulate objects. The character can control the tail’s movements with concentration, but it otherwise tends to move on its own. The tail also improves the character’s balance; apply a +10 to any Physical skill tests where balance is a factor. \textbf{[Low]}

\textbf{Sex Switch:} A complex suite of alterations allows the character to switch their physical sex to male, female, hermaphrodite, or neuter. This change is mentally triggered but takes approximately 1 week to complete. \textbf{[Moderate]}

\textbf{Skin Pocket:} The morph has a pocket within its skin layer, capable of holding and providing concealment (+30) for small items. \textbf{[Trivial]}

\textbf{Temperature Tolerance:} The morph’s temperature regulation and circulation are both substantially enhanced allowing the character to survive in temperatures as low as -30 degrees Celsius and as high as 60 degrees Celsius without discomfort or ill effects. \textbf{[Low]}

\textbf{Toxin Filters:} The morph gains an improved liver and kidneys and biological filters in its lungs. Characters with this augmentation are immune to all chemical and biological toxins, including everything from recreational chemicals to nerve agents to spoiled food. In addition, the character can safely and comfortably breathe smoke and drink salt water. Unlike medichines, toxin immunity prevents the character from experiencing even brief harm or discomfort from a toxin (medichines merely rapidly repair damage caused by the toxin and then remove it from the morph). This augmentation provides no resistance to concentrated acid, nanotechnological attacks, or similar destructive agents. Some characters with this augmentation learn to enjoy the taste of various chemical toxins like cyanide or arsenic. \textbf{[Moderate]}

\textbf{Vacuum Sealing:} To possess this augmentation, the character must also possess some form of bioware armor or carapace armor. The morph has been specially designed to survive the effects of vacuum. The character’s skin resists vacuum as well as protecting the wearer from temperatures from -75 to 100 C. In addition, the character’s mouth, nose, and other orifices can seal sufficiently well to resist vacuum, and the morph possesses a special membrane that extends over their eyes, allowing the character to see in vacuum without risking any eye damage. This augmentation is usually combined with either the enhanced respiration or oxygen storage augmentation, or both together. \textbf{[High]}

\subsubsection{Synthmorphs and bioware}

Though bioware is preferred and more common, many types of bioware can be mimicked with cybernetics. This is especially useful for synthmorphs/ robots, which cannot be enhanced with bioware. The following bioware items may be replicated as cybernetics for synthmorphs and robots:

\begin{itemize}
	\item Chameleon Skin
	\item Drug Glands
	\item Eelware
	\item Emotional Dampers
	\item Enhanced Senses (All)
	\item Grip Pads
	\item Mental Augmentations (All)
	\item Muscle Augmentation
	\item Neurachem
	\item Poison Glands
	\item Prehensile Feet
	\item Prehensile Tail
\end{itemize}

\subsection{Cyberware}
\label{sec:cyberware}

Very little cyberware is physically implanted. Instead, the morph is placed in a healing vat (p. 326) and the vat’s nanobots construct the cyberware inside the biomorph’s body. Cyberware is rarely used for anything that can be accomplished using bioware.

Synthmorphs and bots may also also use cyberware.

\subsubsection{Enhanced senses}

In addition to being able to duplicate the affects of all bioware enhanced senses, there are a few enhanced senses that can only be produced using cyberware.

\textbf{Anti-Glare:} This visual mod eliminates penalties for glare. \textbf{[Low]}

\textbf{Electrical Sense:} The character can sense electric fields. Within 5 meters, the character can instantly tell if an electrical device is on or off and can see the precise location of electrical wiring behind a wall or inside a device. This sense gives the character a +10 modifier on any test involving analyzing, repairing, or modifying electrical equipment. \textbf{[Low]}

\textbf{Radiation sense:} The character can sense the presence and approximate source of all forms of dangerous radiation, including neutrons, charged particles, and cosmic rays. \textbf{[Low]}

\textbf{T-Ray Emitter:} Mounted under the skin of the user’s forehead, this implant generates low-powered beams of terahertz radiation (T-rays) that allow the character to see using reflected T-rays. As discussed in Using Enhanced Senses, p. 302, this implant combined with the enhanced vision enhancement (or a terahertz sensor) allows the user to effectively see through cloth, plastic, wood, masonry, composites, and ceramics as well as being able to determine the composition of various materials. This implant allows the user to see using reflected T-rays for 20 meters in a normal atmosphere and for 100 meters in vacuum. \textbf{[Low]}

\subsubsection{Mental augmentations}

These cybernetic augmentations enhance the brain and mental functions.

\textbf{Access Jacks:} Usually located in the base of the skull or neck, this implant is an external socket with a direct neural interface. It allows the character to establish a direct wired connection using a fiberoptic cable to external devices or other characters, which can be useful in places where wireless links are unreliable or complete privacy is required. Two characters linked via access jack can ``speak'' mind-to-mind and transfer information between their mesh inserts and other implants. All synthmorphs have these by default. \textbf{[Low]}

\textbf{Dead Switch:} This cortical stack (p. 300) accessory is designed to keep the stack from falling into the wrong hands. If the morph is killed, the dead switch wipes and melts the cortical stack completely, so that the ego cannot be recovered. This option is generally only used by covert operatives with recent backups. \textbf{[Low]}

\textbf{Emergency Farcaster:} Only characters with cortical stacks can possess this augmentation. The morph has an implanted quantum farcaster (p. 314) linked to a highly secure storage facility. The high cost of this implant also covers the cost of this storage. Using standard radio and quantum encryption, the farcaster broadcasts full backups of the character’s ego (pulled from the cortical stack) once every 48 hours. At the gamemaster’s discretion, the backup interval may be scheduled more or less frequently, keeping in mind that ego broadcasts are generally limited for security purposes and because they hog bandwidth. These broadcasts only work when the character is in radio contact with the storage facility and is typically only used inside a habitat to broadcast backups back to a nearby space ship. If the radio broadcasts are blocked or jammed, this device cannot make backups.

In the event of a farcaster failure, this augmentation also includes a single-use emergency neutrino broadcaster (p. 314) as well. This broadcaster contains approximately 10 nanograms of antimatter stored in an orange-sized triply-redundant magnetic containment vessel. If the character is dying or urgently wishes to depart the morph, this tiny amount of antimatter is brought into contact with a similarly tiny amount of matter in a controlled fashion that generates a single brief and carefully coded neutrino pulse of the ego’s most recent backup. However, the heat generated by this process literally cooks the entire morph, killing it and destroying all implants and electronics in or on it.

This entire process takes less than 0.1 second and the broadcast can be received as long as the neutrino receiver is within 100 astronomical units of the character. Within the solar system, this implant effectively guarantees the character’s backup. It is less useful on exoplanets where the character is out of neutrino range of their backup facility. The amount of antimatter carried by this implant is sufficiently small enough that it does not produce an explosion and will not damage any surrounding objects. Most habitats carefully scan all visitors to determine if they have this implant and if the amounts of antimatter involved are sufficiently low as not to pose a danger to the habitat and its inhabitants, and some ban this implant entirely. \textbf{[Expensive]}

\textbf{Ghostrider Module:} This implant allows the character to carry another infomorph inside their head. This infomorph could be another muse, an AI, a backed-up ego, or a fork. The module is linked to the character’s mesh inserts, so the ghost-rider can access the mesh. The character may limit the ghostrider’s access, or may allow them direct access to their sensory information, thoughts, communications, and other implants. \textbf{[Low]}

\textbf{Mnemonic Augmentation:} A character with this augmentation and a cortical stack can access digital recordings of all of the sensory data they have experienced in XP format (and they may share these recordings with others). Mnemonic augmentation differs from the eidetic memory bioware because it allows characters to digitally share all of their sensory data with others. It also allows them to closely examine sensory data they did not initially look at. For example, If the character glanced at a note but did not read it, they can later use image enhancement software to enhance this image and in most cases actually read what the note said. Mnemonic augmentation allows the character to clearly hear all background noises, like a conversation at a nearby table that the character only initially heard a few words of. Using mnemonic augmentation to retrieve a specific piece of information is quite easy, but usually requires between 2 and 20 minutes of concentration. \textbf{[Low]}

\textbf{Multi-Tasking:} Only characters with cortical stacks can possess this augmentation. The character has an advanced computer installed in their brain that uses the data in the cortical stack to create several simultaneous short-term forks to handle various mental tasks. By design, this computer automatically reintegrates all of these forks into the character’s core personality after a maximum of 4 hours, earlier if desired. This augmentation allows the character to both plan a speech and engage in intensive mesh-browsing while simultaneously fighting a gun battle or running from pursuit, since each of the forks operates independently. However, these forks can only perform purely mental or on-line interactions. This augmentation can produce a maximum of two forks at a time, giving the character an extra two Complex Actions on every Action Phase for mental or on-line actions. This implant cannot be used simultaneously with any other augmentation that allows for extra actions, or with the mental speed augmentation (p. 308). \textbf{[High]}

\textbf{Puppet Sock:} This implanted computer allows the biomorph’s body (the ``puppet'') to be controlled by another character (the ``puppeteer''). While active, the puppet has no control over their body and is simply along for the ride (at the gamemaster’s discretion, puppets who are tormented by repeated or extensive loss of control may suffer mental stress). The puppeteer may directly ``jam'' the puppet or remote control it in the same way that robots and pods are teleoperated (p. 196). The puppeteer must either be ghost-riding the puppet (see the Ghostrider Module, p. 307) or have a direct communications link (via mesh, radio, laser, etc.). \textbf{[Moderate]}

\subsubsection{Physical augmentations}

This implants enhance the morph’s physical body.

\textbf{Cyberclaws:} The bones on the back of the morph’s hand are bonded to smart material claws. These claws can extend through concealed ports in the morph’s skin and extend 6 inches past the morph’s knuckles. These razor-sharp weapons inflict 1d10 + 3 + (SOM $\div$ 10) damage and have an AP of -2. If combined with eelware (p. 304), they can also inflict electric shocks. Likewise, cyberclaws can also deliver poison or nanotoxins secreted from a poison gland (p. 305) or implanted nanotoxins. \textbf{[Low]}

\textbf{Cyberlimb:} In an age when arms and legs can easily be regrown, many people consider cybernetic prostheses to be vulgar and distasteful. The Scum and others, however, treat them as iconic symbols of self-expression. Standard replacement cyberlimbs function the same as their biological equivalents, though that particular limb receives a +3/+3 Armor bonus when targeted specifically (this bonus does not apply to synthmorphs). Cyberlimbs may be masked to look real (see Synthetic Mask, p. 311), and may also feature small compartments for hiding/storing small objects. \textbf{[Moderate]}

\textbf{Cyberlimb Plus:} More extravagant cyberlimb models are also available, though they require more severe body alteration to accommodate. These limbs apply a +5 SOM bonus per limb (maximum +10). They may be replacement limbs or ``extra'' limbs anchored in the body’s skeletal frame. These cyberlimbs may not be masked. \textbf{[High]}

\textbf{Hand Laser:} The morph has a weapon-grade laser implanted in its forearm, with a flexible waveguide leading to a lens located between the first two knuckles on the morph’s dominant hand. The laser fires from this waveguide, inflicting 2d10 damage with 0 AP. The laser is powered by a small nuclear battery located in the morph’s torso, good for 50 shots before it must be recharged like other beam weapon batteries (p. 338). \textbf{[Moderate]}

\textbf{Hardened Skeleton:} The morph’s skeleton has been laced with strengthening materials. Apply a +5 DUR and +5 SOM bonus. \textbf{[High]}

\textbf{Oxygen Reserve:} The morph has a miniature oxygen tank and rebreather installed in its torso. This implant provides the equivalent of the life support system in a light vacsuit (p. 333), allowing the character to breathe comfortably for up to 3 hours. It feeds oxygen directly to the morph’s blood stream, avoiding problems with pressure changes. Implanted sensors automatically cause the character to use the stored oxygen if they detect poisonous or insufficient atmosphere. Without vacuum sealing, the character can only survive in vacuum for 5 minutes, but remains conscious and active for the entire time, giving them far more time to find shelter or a vacsuit than characters without this implant. For every hour the character is in a breathable atmosphere, this implant recovers one hour of oxygen storage. The implant can be fully recharged within 15 minutes if the character is in a high-pressure mostly oxygen atmosphere. \textbf{[Low]}

\textbf{Reflex Boosters:} The morph’s spinal column and nervous system is rewired with superconducting materials, boosting transmission speed. This raises the character’s REF by +10 and improves Speed by +1. \textbf{[Expensive]}

\subsection{Using enhanced senses}
\label{sec:using-enhanced-senses}

Personal augmentations and technological aids have drastically increased the sensory capabilities of most transhumans. The following notes provide some details on what capabilities these sensory functions provide. The capabilities are typically the same whether it’s a biological sense or a technological sensor, though tech sensors can ``turn off'' certain wavelengths and sense only specific frequencies, whereas biological senses perceive the full spectrum with no ability to filter parts out.

\subsubsection{Sensory databases}

Both technological sensors and enhanced biological senses come equipped with databases of scanned ``signatures'' that make it easier to identify whatever the user is sensing (in the case of bioware, these databases are stored and accessed via the character’s mesh inserts). For example, infrared sensors feature databases listing the heat signatures of different animals and items, making it easier to identify such things. In relevant situations, apply a +20 modifier for identifying targets sensed this way.

\subsubsection{Active vs. passive}

An active scanner must actually emit its particular frequency and then measure the reflections; this means a similar sensor can detect it and home in on the emitting source. For example, a character with enhanced vision can literally see the terahertz radiation emitted by someone using an active terahertz sensor, much like someone with normal vision can see the light emitted by a flashlight.

A passive scanner simply scans frequencies that occur naturally --- there is nothing to give the sensor away.

\subsubsection{Electromagnetic spectrum}

For \textit{Eclipse Phase} rules purposes, the EM spectrum is broken down by wavelength and frequency into these categories: radio, microwave, terahertz, infrared, visible light, ultraviolet, X-rays, and gamma rays.

\textbf{Radar (Radio/Microwave): }Radar sensors work by actively emitting radio waves and microwaves and measuring them as they bounce off the target. Radar works best when detecting metallic objects, and is less effective (-20 modifier) against biomorphs and small items. Resolution is not high, however, so it can see shapes but not colors or fine details. It can be used to detect both speed and movement, can ``see'' through walls (up to a cumulative Armor + Durability of 100), and can detect cybernetic implants or concealed items. At close ranges (1-2 meters), it can detect pulse rate and respiration by measuring the motion of the chest cavity.

\textbf{Terahertz:} Terahertz sensors emit t-rays, measure the reflections, and compare them to a database of terahertz signatures that different items/materials have. The resolution is higher than radar, but with slightly less detail than normal vision. Similar to radar, terahertz sensors can see through walls and other materials, but to a lesser extent (up to a cumulative Armor + Durability of 50). T-rays occur naturally, but terahertz sensors normally require an emitter as they are absorbed by atmosphere (as well as water and metal). In space, however, an emitter would not be required. Likewise, passive terahertz scans within atmosphere have an effective range of 25 meters. T-rays do not penetrate skin, so are ineffective for locating implants.

\textbf{Infrared:} Near-infrared wavelengths are used for night vision, providing resolution and detail equivalent to regular vision under low-light conditions. Mid-long infrared is excellent for detecting heat sources (unobstructed by fog or smoke) and temperature differences (as small as 0.1 degree C), and such thermal imaging will sense the dissipating heat traces left by warm sources on colder ones, allowing the user to see where someone was sitting, trace fading heat footprints, or see what buttons were pressed if they are quick enough. Infrared also detects the blood flow in a biomorph’s face, which can be useful in judging emotional states (+20 modifier to Kinesics Tests), and can spot subsurface implants. Some normally white surfaces are reflective (mirrored) in infrared, potentially allowing an infrared viewer to see around corners or behind themselves. On the other hand, some glass is opaque to infrared light. Infrared is also useful for determining chemical composition (enabling Chemistry Tests by sight alone). Infrared sensory input is passive.

\textbf{Lidar (Visible Light):} Similar to radar, but with much higher resolution, lidar actively bounces light from the infrared through ultraviolet spectrum off a target and measures the backscatter, fluorescence, and other properties. Lidar is very useful for detecting atmospheric chemical properties and weather. Like radar, it can be used to measure a target’s range and speed, or develop a three-dimensional image. One clever use of lidar is to precisely ``map'' the position of everything in a room (taking several turns of scanning) and then check that positioning later to see if anything has been moved.

\textbf{Ultraviolet:} Some objects are fluorescent in ultraviolet light, including some animals, flowers, insects, urine, and minerals (which show up much better in ultraviolet than regular light). Some plants and animals have patterns that can only be seen in ultraviolet. Likewise, chemical dyes that only show up under ultraviolet, or that make certain substances (like blood) fluoresce under ultraviolet light, have various security purposes. Some glass is opaque at ultraviolet wavelengths.

\textbf{X-Ray/Gamma-Ray:} Backscatter imaging systems using X- and gamma-ray frequencies produce high-resolution three-dimensional images and are very useful for detecting concealed weapons and implants. Such imagers are very good at penetrating walls and metal (up to a cumulative Armor + Durability of 200, at least at levels safe to transhumans). These sensors can, of course, also detect the presence of harmful radiation.

\subsubsection{Soundwaves}

The transmission of vibrations through a medium, sound is broken down into infrasound (frequencies below standard human hearing), normal acoustic range, and ultrasound (frequencies above standard human hearing). Soundwaves do not propagate in vacuum.

\textbf{Ultrasound:} Ultrasound sonar operates much like radar, bouncing sound waves off a target and measuring the returning echoes. Ultrasound imaging is similarly low-resolution, showing shapes and movement but no colors and few details unless measured closely (1-2 meters). Ultrasound is good for identifying how dense a material is, however, can detect denser materials hidden beneath less dense ones. Many medical devices utilize ultrasound, and ultrasound sensors can also detect gas leaks, frictional motor noises, and similar mechanical emissions. Ultrasound sensors are typically unaffected by noise clutter from standard acoustic frequencies.

\textbf{Infrasound:} Infrasound travels much further than regular sound frequencies (hundreds of kilometers). Mechanical machinery, seismic disturbances, tornados, explosions, waterfalls, and certain weather phenomena create infrasound waves. Large animals such as elephants and whales use infrasound to communicate via the ground over large distances, though infrasound data transfer is too slow for complex communications.

\subsubsection{Combined sensor systems}

When used in combination, these sensor technologies can be potent. For example, the use lidar, thermal imaging, and radar can provide a threedimensional map of a building and everyone and everything inside.


\subsection{Nanoware}
\label{sec:nanoware}

All augmentation nanoware is advanced nanotechnology (p. 328), consisting of a grape-sized nanobot generator that produces specialized nanomachines. Nanoware is available for synthmorphs and bots in addition to biomorphs.

\textbf{Implanted Nanotoxins:} The morph has an implanted nanobot hive that produces nanotoxins (p. 324). This implant is designed so that the character can deploy these nanobots instantly via a scratch with claws, spraying with saliva, or simply making continuous bare-skin contact. Characters can choose whether or not to deploy these nanobots. Each nanotoxin generator only produces a single variety of nanobots, with the most common types being ones designed to kill or incapacitate almost any living target or ones designed to destroy delicate machinery. Characters are immune to their own nanotoxins. Nanotoxins are highly restricted and many habitats will not allow anyone with this implant on board. \textbf{[Moderate]}

\textbf{Medichines:} This is the most common form of nanoware. These nanobots monitor the user’s body at a cellular level and fix any problems that arise. Medichines eliminate most diseases, drugs, and toxins (but not nanodrugs or nanotoxins) before they can do more than minor harm to the host (see Drug Effects, p. 318). If desired, the user can temporarily override this protection to permit intoxication or other effects, but unless the character activates a second specially labeled override, medichines prevent the toxins from accumulating to lethal or permanently harmful levels. In this case, they can also be activated at a later point to reduce a drug or toxin’s remaining duration by half.

Medichines allow the character to ignore the effects of 1 wound. They also speed normal healing as noted under Biomorph Healing, p. 208. If the user suffers 5 or more wounds at once, or more than 6 wounds in an hour, the damage has exceeded the medichines’ ability to repair. In this case, the medichines place the character into a medical stasis, where their mind and body are perfectly preserved, but where the character cannot act in any way. Under these circumstances the medichines also send out a priority call for emergency services via the character’s mesh inserts. Medichines for synthmorphs and bots consist of nanobots that monitor and repair the shell’s integrity and internal system functions. Note that the synthmorph version of medichines allows the synthmorph to self-repair in the same manner by which a biomorph with medichines would naturally heal (p. 208). \textbf{[Low]}

\textbf{Mental Speed:} With this nanoware system, nanobots alter the character’s neural architecture and augment the functioning of their neurons. The character can deliberately speed up their mind to think and also receive and process sensory information far faster than ordinary humans. Time seems to subjectively slow down for the character, allowing them to carefully plan their next action, even if they only have a split second to do so. With this system active, the character can discern things occurring too fast for a normal human to perceive, such as the individual frames of an old analog film or understanding sounds that were accelerated to many times their normal speed. The character can also read 10 times faster than normal and can track the paths of bullets and similar fast-moving objects with a successful Perception Test.

When using this augmentation, the character gains two extra Complex Actions during each Action Phase that may only be used for mental actions. The character also receives a +30 Initiative bonus. The character thinks at normal speed whenever this nanoware is inactive. This nanoware is incompatible with any other augmentation that provides any form of extra actions, such as multi-tasking. This augmentation can be used as often as desired, but actively using it renders ordinary conversation and social interactions difficult and requires concentration to maintain. \textbf{[High]}

\textbf{Nanophages:} These nanobots patrol the body, alert for signs of intrusive nanodrugs or -toxins and destroying them before they have more than a minor effect. Nanophages provide automatic immunity against nanodrugs and nanotoxins unless they are specifically commanded to stand down by the user, via their mesh implants. \textbf{[Moderate]}

\textbf{Oracles:} These neural macrosensing nanobots pay attention to the sensory input on which the character is not focusing, alerting them about important things they might otherwise overlook. Oracles also act as a sort of memory buffer and search aid, extending short term memory, helping the character recall memories and details, and crosschecking them with other memories. Oracles negate Perception modifiers for distraction, apply a +10 modifier to Investigation Tests, and add a +30 bonus to memory-related tests. \textbf{[Moderate]}

\textbf{Respirocytes:} These nanobots act as highly-efficient artificial red blood cells, increasing the ability to transfer oxygen and carbon dioxide. This increases the morph’s ability to hold their breath to 4 hours and increases DUR by +5. \textbf{[Moderate]}

\textbf{Skillware:} The morph’s brain is laced with a network of artificial neurons that may be formatted with downloaded information. This allows the user to download skillsofts (p. 332) into their brains, gaining the use of those programmed skills until the skillsoft is erased or replaced. Skillware systems are only capable of handling 100 total skill points worth of skillsofts at a time. \textbf{[High]}

\textbf{Skinflex:} This disguise implant allows the user to restructure their facial features and musculature and alter skin tone and hair color. The entire process takes a mere 20 minutes. Skinflex adds +30 to Disguise Tests. \textbf{[Moderate]}

\textbf{Skinlink:} Skinlink nanobots live on the morph’s external skin or shell, automatically swarming over and creating a physical connection with any electronics the user touches. They also take advantage of the electrical field in a biomorph’s skin for communication. They allow the user to communicate and mesh with any devices merely by touching them. This is considered a wired link, and so is not subject to wireless interception or interference. Two skinlinked characters can also communicate and mesh simply by touching. \textbf{[Moderate]}

\textbf{Wrist-Mounted Tools:} The morph has a 6 centimeter- wide metal band containing nanobot generators implanted around each wrist. These nanobots link together to duplicate the function of a utilitool (p. 326), creating narrow, highly flexible arms that each ends in a specialized tool. These nanobots can also produce tiny fiber optics to allow the character to see through small openings, as well as being able to create small weapons equal to bioware claws. The fact that these tool are mentally controlled gives the character a +20 modifier to skills involving repairing or modifying devices with mechanical parts, opening locks or disarming alarm systems, or performing first aid. \textbf{[Moderate]}


\subsection{Cosmetic mods}
\label{sec:cosmetic-mods}

In an age of universal beauty, artistic cosmetic modification of your body is commonly pursued by many transhumans. Body mods once considered dangerous or edgy are now safe and commonplace, especially among factions like the anarchists, scum, or brinkers.

\textbf{Bodysculpting:} If your morph’s enhanced physique isn’t enough, you can take it further with custom bodysculpting such as as elongated ears or fingers, nose alteration, hair addition/removal, feathers, exotic eyes, snakeskin, endowed genitalia, and more unusual physical alterations. \textbf{[Low]}

\textbf{Nanotats:} Tattoos created with nanobots can move around the body, change shape/color/brightness, texture, alternate text and images, and/or even create minor holographic effects on the skin’s surface, all controllable via mesh inserts. \textbf{[Low]}

\textbf{Piercings:} Name any part of the body and someone’s figured out a way to pierce it, probably multiple times. Hoops, barbells, plugs, and chains are extremely common, often made of shapechanging smart materials. \textbf{[Trivial]}

\textbf{Scarification:} Given modern medical abilities, scars of any sort are purely an affectation. \textbf{[Trivial]}

\textbf{Scent Alteration:} Minor changes to a body’s biochemistry can alter a character’s natural smell or constantly perfume them. \textbf{[Low]}

\textbf{Skindyes:} Dye jobs are available in all conceivable colors and patterns. \textbf{[Trivial]}

\textbf{Subdermal Implants:} Adding small implants under the skin can create bumps, ridges, piercing anchors, and similar textures and alterations. \textbf{[Trivial]}


\subsection{Robotic enhancements}
\label{sec:robotic-enhancements}

The following modifications are only available to synthmorphs/robots.

\subsubsection{Armor}

These armor modifications replace the synthmorph’s built-in Armor rating.

\textbf{Heavy Combat Armor:} The synthmorph’s frame is loaded with armor that offers protection from heavy weapons for serious combat operations. This modification is bulky and noticeable; the bot frame is encased in a heavy-duty carapace. It increases the bot’s built-in Armor to 16/16. The shell’s mobility systems and power output are also enhanced to deal with the extra load. \textbf{[High]}

\textbf{Industrial Armor:} The shell is equipped with protection against collisions, extreme weather, industrial accidents, and similar wear-and-tear. Increase the bot’s built-in Armor rating to 10/10. [Moderate] Light Combat Armor: The synthmorph’s frame is protected by armor designed for policing and security duties. This increases the bot’s built-in Armor to 14/12. \textbf{[Moderate]}

\subsubsection{Mobility systems}

Shells are designed with a wide-range of propulsion systems, and are sometimes built for a specific environment/ gravity. Some synthmorphs may have multiple mobility systems. Many such systems are retractable, meaning they can be folded away into the shell’s frame.

\textbf{Hopper:} Hoppers have two or more legs designed to propel the morph forward or up, much like a frog or grasshopper. \textbf{[Moderate]}

\textbf{Hovercraft:} The shell uses an impeller to blast a cushion of high-pressure air off the surface below, repelling the frame off the ground (modern hovercraft do not use rubber skirts). Most hovercraft travel a meter or so above the ground, but can temporarily levitate themselves higher for short periods. \textbf{[Low]}

\textbf{Ionic:} The shell uses principles of magnetohydrodynamics to levitate and fly, by ionizing surrounding air into plasma to create lift and momentum. The shell is also spun for stability. This system does not work in vacuum, but an underwater version uses the same mechanics for propulsion in liquid environments. \textbf{[High]}

\textbf{Microlight:} Popular in low-grav and microgravity environments, microlights encompass several types of ultralight or lighter-than-air systems, such as powered paragliders, autogyros, balloons, aerostats, and blimps. These systems do not work in vacuum. [Low] Roller: Only for circular shells, this system allows the synthmorph to roll like a ball. The shell rolls around an interior axle, propelled by a motor-driven pendulum. \textbf{[Moderate]}

\textbf{Rotorcraft:} Rotating blades create lift, allowing the shell to move and hover like a helicopter. Most models use tilt-rotors or tilt-wings so that the rotorblades may be moved forward (for faster propellerlike propulsion) and for better maneuverability in zero-G. This system does not work in vacuum. \textbf{[Low]}

\textbf{Snake:} Commonly used by slitheroids, these shells use lateral undulation, flexing their body from left to right and waving their frame forward. Such shells may also use sidewinding or a concertina motion (straightening forward, then retracting the rear) to move. They also featured gyroscope stabilization so that they may circle into a hoop and roll like a wheel. [Moderate] Submarine: Designed for undersea mobility, submarine shells use propellers or pumpjets to push through water. \textbf{[Moderate]}

\textbf{Tracked:} Tracked shells use smart rotating treads to work their way across surfaces that would bog down other ground vehicles. They can prop themselves up in order to overcome taller obstacles or to lay themselves down to bridge across a ditch or crevice. \textbf{[Low]}

\textbf{Thrust Vector:} These shells use either turbofans or turbojets to create atmospheric lift with a set of wings. The engines may be maneuvered to point and generate thrust in different directions for vertical takeoffs/landings and better maneuverability in zero-G. \textbf{[Moderate]}

\textbf{Walker:} Walkers use two or more limbs to walk or crawl across a surface. Many use grip pads (p. 305) or magnetic systems (p. 310) to stick to surfaces. \textbf{[Low]}

\textbf{Wheeled:} Most wheeled shells feature smart spokes that allow the wheels to conform their shape to obstacles and even climb stairs. Some low-grav shells feature puncture-resistant and self-repairing compressed-gas tires. \textbf{[Low]}

\textbf{Winged:} Primarily used by smaller shells, this system of four independently-controlled wings allows the shell to hover or move rapidly in any direction. \textbf{[Low]}

\subsubsection{Physical modifications}

These mods are applied to the shell’s physical frame.

\textbf{Extra Limbs:} The shell is equipped with one or more extra limbs. A character using these limbs suffers an off-hand modifier (p. 193). These limbs may be arms (with hand/grippers/etc.), legs, tentaclelike, or otherwise articulated and/or prehensile. Some shells have rotational frames that allow them to move limbs around their body. \textbf{[Low]}

\textbf{Fractal Digits:} The synthmorph has ``bush robot'' digits that are capable of splitting into smaller digits, and those smaller digits into micro digits, and so on down to the micrometer scale, allowing for ultra-fine manipulation. Apply a +20 COO modifier where such fine manipulation is a factor (such as detailed repair work). The bot must have functioning nanoscopic vision (p. 311) to get this bonus. \textbf{[Moderate]}

\textbf{Hidden Compartment:} The shell has a concealed aperture for a shielded interior compartment, ideal for storing valuables or smuggling contraband. Apply a $-$30 modifier to detect this compartment either manually or with sensor scans. \textbf{[Low]}

\textbf{Magnetic System:} A magnetic system allows the shell to cling to most ferrous materials. This enables the character to walk in zero-G situations by magnetically adhering surfaces, hang upside down, and hold onto devices without letting them drop or drift away. The shell receives a +30 modifier whenever maintaining a magnetic hold on something. \textbf{[Low]}

\textbf{Modular Design:} This shell is designed to lock together with similar modular morphs in different architectural patterns to create larger gestalt forms. When united with other modules, the group is treated as a single unit/morph, with shared capabilities. If damaged and then separated, damage and wounds are distributed evenly between modules; uneven amounts are allocated randomly. The exact capabilities of different shapes depends on the composition, and is largely left in the gamemaster’s hands. \textbf{[High]}

\textbf{Pneumatic Limbs:} The limbs are equipped with pneumatic cylinder systems that can generate up to 1,500 pounds of thrust. This allows the shell to push off and make impressive jumps (a synth of human size/weight can leap over 2 meters up). Apply a +20 to Freerunning Tests. A pneumatic limb used to strike an opponent in unarmed combat inflicts an extra 1d10 damage. \textbf{[Low]}

\textbf{Retracting/Telescoping Limbs:} The shell’s limbs can either be retracted completely inside it’s frame and/or extended for extra length (usually up to 1 or 2 meters extra). Telescoping limbs may give the shell a reach advantage in melee combat (p. 204). \textbf{[Low]}

\textbf{Shape Adjusting:} This shell is made from smart materials that allow it to alter its shape, altering its height, width, circumference, and external features, while retaining the same mass. This modification is typically employed to reshape the morph into special configurations adapted to specific tasks (for example, lengthening to crawl through a tunnel, widening its base for stability, expanding to reach out and attach to multiple access point simultaneously, and so on). This mod also allows the morph to change its features for disguise purposes; apply a +30 modifier to Disguise Tests. \textbf{[High]}

\textbf{Structural Enhancement:} This modification bolsters the shell’s structural integrity, boosting its ability to take damage. Increase Durability by 10 and Wound Threshold by 2. \textbf{[Moderate]}

\textbf{Swarm Composition:} The shell is not a single unit but a swarm of hundreds of insect-sized robotic microdrones. Each individual ``bug'' is capable of crawling, rolling, hopping several meters, or using nanocopter fan blades for airlift. The cyberbrain, sensor systems, and implants are distributed throughout the swarm. Though the swarm can ``meld'' together into a roughly child-sized shape, the swarm is incapable of tackling physical tasks like grabbing, lifting, or holding as a unit. Individual bugs, however, are quite capable of interfacing with electronics. Swarms cannot carry most gear or wear armor, and may not make strength-based SOM-linked skill tests. For combat purposes, use the same rules as given for nanoswarms, p. 328. Damage and wounds are reflected as damaged/ massacred bugs. The swarm may be ``healed'' by manufacturing more bugs. \textbf{[High]}

\textbf{Synthetic Mask:} The synthmorph is equipped with a realistic outer casing of faux-skin and carefully sculpted to pass as a biomorph (perhaps even a particular person). The morph can cry, spit, have sex, and will even bleed if cut. Only a detailed physical examination or a radar, terahertz, or x-ray scan will detect the synthmorph’s true nature, and even then such exams/scans suffer a $-$30 modifier. [Moderate] Weapon Mount: The shell carries a built-in (or builton) weapon. This weapon mount may be either internal (concealed, only weapons small in relation to the shell may fit, $-$30 to Perception Tests to detect) or external (visible). It may be fixed (one direction only), swiveling (limited field of fire), or on an articulated mount (all directions). \textbf{[Low; Moderate for concealed/articulated]}

\subsubsection{Sensors}

\textbf{360$^{\circ}$ Vision:} The shell’s visual sensors are situated for a 360-degree field of vision. \textbf{[Low]}

\textbf{Chemical Sniffer:} This sensor detects molecules in the air and analyzes their chemical composition. It enables Chemistry Tests to determine the presence of gases, including toxins and other fumes. It can also detect the presence of explosives and firearms. \textbf{[Moderate]}

\textbf{Lidar:} This sensor emits laser light and measures the reflections to judge range, speed, and image the target. See Using Enhanced Senses, p. 302. \textbf{[Low]}

\textbf{Nanoscopic Vision:} The shell’s visual sensors can focus like a microscope, using advanced superlens techniques to beat the optical diffraction limit and image objects as small as a nanometer. This allows the character to view and analyze objects as small as blood cells and even individual nanobots. The synthmorph must stay relatively steady to view objects at this scale. \textbf{[Moderate]}

\textbf{Radar:} This sensor system bounces radio or microwaves off targets and measures the reflected waves to judge size, composition, and motion. See Using Enhanced Senses, p. 302. \textbf{[Low]}

\subsection{Armor}
\label{sec:armor}

Modern personal armor systems have advanced from the high modulus polyethylene thermoplastics and aramid fabrics of the early 21st century. Armor in \emph{Eclipse Phase} is derived from biotech, in the form of organoweave fibers and crystalline-grown plates, and nanotech, in the form of shock-absorbing fullerene (p. 298) materials. Occasionally other materials are used, such as metallic glass plates or shear-resistant fluids that harden against impacts. Such armor protects against (armor-piercing) bullets and kinetic impacts as well as bladed weapons and piercing sharp objects. They also insulate against both the explosive heating of energy weapons and electrical shocks. While such armor protects against bullets, the layers of material catch the bullet and redistribute its kinetic energy across the body, which can still result in severe blunt force trauma.

Rules for armor in combat can be found on p. 194. Armored exoskeletons are listed on p. 343.

\textbf{Armor Clothing:} The extra-resilient organoweave fibers and fullerene materials that offer basic protection against kinetic and energy weapons can be woven in with normal smart materials to create a wide range of discreet armor clothing that provides a subtle level of security. Such protective garments are indistinguishable from regular clothing and come in all styles and designs. Armor clothing provides an Armor Value of 3/4. \textbf{[Trivial]}

\textbf{Armor Vest:} Armor vests provide more thorough protection to a body’s vital areas, covering the abdomen and torso completely, protecting the neck with a rigid collar, and even providing wrap-under protection for the groin. Though armor vests are not bulky, they are obvious as armor. Armor vests may be worn with armor clothing without penalty. Armor vests provide an Armor Value of 6/6. \textbf{[Low]}

\textbf{Body Armor (Light):} These high performance armor outfits protect the wearer from head to toe. An integrated armor vest is supplemented with increased protection on the limbs and joints, while still managing to be flexible and non-restrictive. Body armor is typically worn by security and police forces, and supplemented with a helmet. It provides an Armor Value of 10/10. \textbf{[Low]}

\textbf{Body Armor (Heavy):} Similar to light body armor, but with extra protective layers, often ergonomically manufactured to conform to a specific character’s body, and an environmental seal with climate control to protect the wearer from hostile environments. It provides an Armor Value of 13/13. \textbf{[Moderate]}

\textbf{Crash Suit:} Designed for both industrial worksite safety and protection from accidental zero-G collisions, crash suits are also favored by sports enthusiasts and explorers. The basic jumpsuit offers comfortable protection equal to that of armor clothing. When activated with an electronic signal, however, elastic polymers within the suit stiffen and form rigid impact protection for vital areas. Crash suits provide an Armor Value of 3/4 when inactive and 4/6 when activated. \textbf{[Low]}

\textbf{Helmet:} This armor accessory is usually worn with body armor or a battle suit. Light helmets are open, whereas full helmets latch on and provide an environmental seal with a 12 hour supply of air. Light helmets provide an Armor Value bonus of +2/+2, whereas full helmets add +3/+3. Helmets are often equipped with an ecto (p. 325), a radio booster (p. 313), and sensors equal to specs (see p. 325). \textbf{[Trivial]}

\textbf{Riot Shield:} Used for mob suppression, riots shields are light-weight, tough, and may be set to electrify on command, stunning anyone who comes into contact with the outer surface (treat as shock glove effects, p. 334). Riot shields provide an Armor Value bonus of +3/+2. \textbf{[Low]}

\textbf{Second Skin:} This lightweight bodysuit, woven from spider silks and fullerenes, is typically worn as an underlayer, though some athletes use it as a uniform. It provides minimal protection, but may be worn with other armor without penalty. It provides an Armor Value of 1/3. \textbf{[Low]}

\textbf{Smart Skin:} Smart skin is an advanced nanofluid that covers the wearer’s skin. It resembles liquid mercury but retains the texture and flexibility of normal skin until activated, at which point the material becomes rigid enough to protect the wearer and distribute the kinetic energy (though still flexible enough at the joints not to impede movement). A specialized hive, worn by the character, replenishes the nanobots and stores them when not in use. Deploying the nanobots across the body takes a full Action Turn. Smart skin has an Armor Value of 3/2, and may be worn with other armor without penalty. \textbf{[Low]}

\textbf{Spray Armor:} This fast armor application comes in a spray can and disperses a smart chemical polymer that sticks to bare flesh (but does not adhere to hair and eyes). The polymer solidifies into a form fitting body armor fabric when exposed to body temperature with the look and feel of a latex suit. Spray armor does not work on synthetic morphs or on clothing or other armor. The color and feel of the armor can be adjusted with electric currents and additional polymers, making it popular among some socialite and nightlife scenes. The spray-on armor does not wash off, but degrades 1 point of armor (both energy and kinetic) every 12 hours. It may be removed with a special nanotech solvent. Spray armor has an Armor Value of 2/2. \textbf{[Low]}

\begin{table}
\begin{tabular}{|l|r|r|r|}
\hline
\multicolumn{2}{|c|}{\textbf{Armor values}}			\\
\hline
\textbf{Armor}					& \textbf{Energy} & \textbf{Kinetic}	& \textbf{Page} \\
\hline
Armor Clothing					& 3				& 4				& 311	\\
\hline
Armor Vest					& 6				& 6				& 312	\\
\hline
Battle Suit Powered Exoskeleton	& 18				& 18				& 344	\\
\hline
Bioweave Armor (Light)			& 2				& 3				& 302	\\
\hline
Bioweave Armor (Heavy)			& 3				& 4				& 302	\\
\hline
Body Armor (Light)				& 10				& 10				& 312	\\
\hline
Body Armor (Heavy)				& 13				& 13				& 312	\\
\hline
Carapace Armor					& 11				& 11				& 303	\\
\hline
Crash Suit (Inactive)			& 3				& 4				& 312	\\
\hline
Crash Suit (Active)				& 4				& 6				& 312	\\
\hline
Exowalker						& 2				& 4				& 344	\\
\hline
Hard Suit						& 15				& 15				& 334	\\
\hline
Helmet (Light)					& +2				& +2				& 312	\\
\hline
Helmet (Full)					& +3				& +3				& 312	\\
\hline
Hyperdense Exoskeleton			& 6				& 12				& 344	\\
\hline
Riot Shield					& +3				& +2				& 312	\\
\hline
Second Skin					& 1				& 3				& 312	\\
\hline
Smart Skin					& 3				& 2				& 312	\\
\hline
Smart Vac Clothing				& 2				& 4				& 325	\\
\hline
Spray Armor					& 2				& 2				& 312	\\
\hline
Synthmorph Industrial Armor		& 10				& 10				& 310	\\
\hline
Synthmorph Combat Armor (Light)	& 14				& 12				& 310	\\
\hline
Synthmorph Combat Armor (Heavy)	& 16				& 16				& 310	\\
\hline
Transporter Exoskeleton			& 2				& 4				& 344	\\
\hline
Trike Exoskeleton				& 2				& 4				& 344	\\
\hline
Vacsuit (Light)				& 5				& 5				& 333	\\
\hline
Vacsuit (Standard)				& 7				& 7				& 333	\\
\hline
\end{tabular}
\label{tab:armor-values}
\end{table}


\subsection{Armor mods}
\label{sec:armor-mods}

Armor modifications add extra materials or coatings that either enhance the armor’s resistance to certain dangers or provide other effects. Armor mods may be easily added or removed with the appropriate nanobot applicators.

\textbf{Ablative Patches:} These thin and light slap-on patches of stick to armor and are designed to absorb heat and energy from beams and explosions, safely vaporizing and blowing hot gas away. Ablative patches increases the Armor Value by +4/+2, but each hit reduces both the energy and kinetic value of the ablative armor by 1. \textbf{[Trivial]}

\textbf{Chameleon Coating:} This provides the armor with the same effect as the chameleon cloak (p. 315). \textbf{[Trivial]}

\textbf{Fireproofing:} Fireproofing includes the addition of heat-resistant ceramic or fire-resistant layers, both capable of withstanding extremely high temperatures. Fireproofing increases the Armor Value by +2/+0, and provides an additional 10 points of armor against heat or fire specifically. \textbf{[Trivial]}

\textbf{Immunogenic System:} The immunogenic mod adds an active nanobot swarm, maintained by a specialized hive, that coats the outer layer of armor and also the non-armored parts of the wearer’s morph. It acts as an outer immune system designed to neutralize toxic agents and nanotoxins with which it comes into contact. This provides immunity to drugs, toxins, and nanotoxins applied dermally, such as with a slap patch or splash grenade. It has no effect on inhaled, oral, or injected drugs (including coated weapons). \textbf{[Low]}

\textbf{Lotus Coating:} The armor has been impregnated with a superhydrophic coating (contact angle of around 170$^{\circ}$) that repels all water-like liquids. If the armor is splashed by liquid toxins or chemicals, the effect is reduced since the liquids starts to roll off the armor. Apply a +30 modifier when defending against liquid-based attacks. \textbf{[Trivial]}

Offensive Armor: When activated, the outer layer of this armor is rigged to shock anyone or anything that contacts it with electricity. Treat its DV and effect as a shock baton (p. 334). \textbf{[Low]}

\textbf{Reactive Coating:} A thick layer of advanced nanotech is applied to the armor, protecting it with a colony of nanobots designed to sense incoming attacks. When an attack strikes the coating, it detonates to disrupt the attack. Bursts and full autofire are treated as a single attack. A reactive coating increases the Armor Value by +5/+5, but each detonation automatically inflicts 1 point of damage on the wearer. Reactive armor also works against melee attacks, but the attacker also suffers 1d10 $\div$ 2 (round up) points of damage per attack (armor protects) from the microexplosion. Reactive coating only works against 5 attacks, after which the specialized nanobot hive replenishes the coating at the rate of 1 use per hour. \textbf{[Moderate]}

\textbf{Refractive Glazing:} A combination of reflectors, refractive metamaterials, and an energy transfer system with heat radiators provides extra protection against energy weapons. Increase the Armor Value by +3/+0. \textbf{[Low]}

\textbf{Self-Healing:} The armor is equipped with a nanohive that acts like repair spray (p. 333). \textbf{[Moderate]}

\textbf{Shock Proof:} Shock proof armor is electronically insulated to discharge and reduce the effect of shock weapons. Apply an additional +10 modifier when resisting the DV and effects of shock weapons (p. 204). \textbf{[Low]}

\textbf{Thermal Dampening:} Thermal dampening obfuscates heat signatures by converting body heat into electric energy. It makes the target more difficult to spot with thermal sensors; apply a $-$30 modifier for Perception Tests. \textbf{[Moderate]}

\subsection{Communications}
\label{sec:communications}

The oldest and most widespread communications technology still in regular use is radio. Every habitat and world inhabited by transhumanity is awash in radio traffic, with humans, machines, and uplifts all constantly communicating with one another. The smallest radios are no larger than a spec of dust and have a range of no more than 20 meters, while the largest are the size of a truck and have a range of many thousands of miles. Radios large and small are ubiquitous and almost all devices contain at least short-range radios so they may interact with the mesh. Most morphs are equipped with basic mesh inserts (p. 300) that include an implanted radio. For radio ranges, see p. 296.

\textbf{Fiberoptic Cable:} Fiberoptic cables are used to establish wired connections between two devices. Given the ubiquity of radios and the tangled mess wires cause, they are typically only used for privacy (unlike radio communication, fiberoptic signals may not be intercepted) or in areas with heavy radio interference. \textbf{[Trivial]}

\textbf{Laser/Microwave Link:} These portable devices are used to establish a tight-beam, line-of-sight communications channel with another laser or microwave link. The range of these transceivers varies widely with environmental factors, but approximates 50 kilometers in atmosphere and 500 kilometers in space (though horizon limits must be kept in mind, being 5 kilometers at ground level on Earth and less on smaller bodies). Lasers are subject to interference from fog, dirt, smoke, and similar visual chaff, while microwaves may be hindered by metallic obstructions. These links may only be intercepted by getting directly in between the beams. Some teams carry a micro version of this system, worn on their person, allowing line of sight intra-team communications that cannot be intercepted like radio. \textbf{[Moderate]}

\textbf{Radio Booster:} This device boosts the range and sensitivity of short-range radios, like those from implants, ectos, or microbugs. The booster must be with the shorter-ranged device’s range (or directly linked via fiberoptic cable). It will repeat any transmissions received from that device, but at its extended range of 25 kilometers in urban areas (250 kilometers remote areas). Broadcasts from a radio booster are easy to receive by anyone looking for broadcasts (see Wireless Scanning, p. 251), though transmissions may be stealthed (p. 252). Boosters are commonly used by characters traveling far from habitats or other civilized regions. \textbf{[Low]}


\subsection{Neutrino communicators}
\label{sec:neutrino-communicators}

Neutrinos are particles that can pass through any solid matter with ease and are impossible to block. As a result, they make an ideal medium for communications. Unfortunately, they are also easy to intercept. Even a tight beam of neutrinos sent between two locations can be intercepted simply by placing another receiver behind the location the broadcaster is sending to. Neutrino communicators require a large power plant to power the high energy particle interactions required to generate the neutrino broadcast. Neutrino receivers are also relatively large, with the smallest occupying 100 cubic meters. In most cases, neutrino communicators are designed to broadcast neutrinos in all directions, though tight-beam transmissions are also possible. Quite often neutrino communications take advantage of quantum farcasting for security.

\textbf{Neutrino Transceiver:} This transceiver is capable of generating and receiving neutrino signals at a range of at least 100 astronomical units. It is large, with a size of 8 cubic meters (in a cube 2 meters on a side), but they can be loaded onto large vehicles. To function, it must be connected to a large power plant, such as one found in habitats or large spacecraft. The cost and size of this device includes the computer necessary for quantum farcasting. \textbf{[Expensive]}

\subsection{Quantum farcasters}
\label{sec:quantum-farcasters}

Quantum farcasters are special computers designed to protect a communications channel (such as fiberoptic, radio, laser/microwave, or neutrino) with unbreakable encryption. To function, two or more quantum farcaster computers must first be entangled together (on a quantum level) in the same physical location. The farcasters may then be separated, at which point they may continue to exchange encrypted data via quantum teleportation. This data exchange requires a standard communications link (fiberoptic, radio, laser/microwave, or neutrino), and so is limited by the speed of light, but it is a high bandwidth form of communications. The quantum encryption used by these entangled farcasters is unbreakable, and any attempted interception is immediately detected and neutralized. A quantum farcaster may not be used to securely communicate with any farcasters other than the ones it is entangled with.

Because it is exceptionally safe and secure, quantum farcasting via neutrino communications is the primary means of both long-distance communication between habitats and egocasting (p. 276). The neutrino signal cannot be blocked and it can only be decrypted if a character has access to the computer that is sending or receiving the signal.

\textbf{Miniature Radio Farcaster:} Miniature farcasters communicate with each other using standard radio transceivers. As noted above, they may only securely communicate with the other farcasters with which they are entangled. Most miniature farcasters are worn as jewelry or fitted into clothing or other equipment. \textbf{[Low]}


\subsection{Quantum entanglement communication}
\label{sec:quantum-entanglement-communication}

The rarest form of communications is quantum entangled (QE) communication. QE communication is instantaneous and works over any distance, but is also very limited. QE communication requires pairs of entangled particles known as qubits. To use QE, large number of pairs of qubits are created and then separated from each other. Millions of these separated pairs of particles are stored in special containers known as qubit reservoirs. If two QE communicators each have a qubit reservoir containing qubits that are each entangled with qubits in the other communicator’s qubit reservoir, then characters can use the two QE communicators to commutate with one another instantaneously. Characters can use QE to instantly communicate between any two locations, even if one character is in the solar system and the other has passed through a Pandora gate and is standing on a planet 500 light years away.

Each bit of data transmitted between these two QE comms uses up one qubit. Once all of the qubits are used up, the two QE comms can no longer communicate with each other until they each get a new batch of entangled qubits. Qubits are expensive to produce, contain, and transport, making this an exceedingly expensive form of communication. As a result, extremely high bandwidth communications like full sensory AR and egocasting cannot be performed using QE communication.

\textbf{Portable QE Comm:} This is a handheld FTL communications device. The actual communications unit can be made as small as desired, but must be large enough to connect to or hold a qubit reservoir. Because qubit reservoirs are relatively large and must be replaced, they are rarely implanted. Some miniature farcasters are designed so that users can also attach qubit reservoirs to enable them to be used for both light speed and FTL communication. \textbf{[Low]}

\textbf{Low-Capacity Qubit Reservoir:} Low-capacity qubit reservoirs can be used for 10 hours of high-resolution video conferencing or meshbrowsing and 100 hours of voice or text only communications. \textbf{[High]}

High-Capacity Qubit Reservoir: High-capacity qubit reservoirs can be used for 100 hours of high-resolution video conferencing or meshbrowsing and 1,000 hours of voice or text only communications. \textbf{[Expensive]}


\subsection{Covert and espionage technologies}
\label{sec:covert-espionage-tech}

These technologies allow characters to acquire protected information and to gain access to places that others try to keep them out of. Many of these devices are mesh-capable and equipped with radios, see p. 296 for radio ranges.

\textbf{Chameleon Cloak:} This loose, poncho-like cloak contains a network of sensors that perceive wavelengths from microwave to ultra-violet. A similar network of miniature emitters precisely replicate the information its sensors receive, making the wearer seem transparent to those wavelengths. A chameleon cloak allows a character to effectively become invisible as long as they are stationary or not moving faster than a slow walk. When worn by someone moving faster, the cloak still provides a +30 modifier to Infiltration Tests to avoid being seen or noticed.

Chameleon cloaks are not effective against radar, x-ray, or gamma-ray sensors. They do hide the character from thermal infrared, however, by absorbing the character’s body heat into its heat sink. The cloak can only absorb a character’s body heat for one hour before it must emit this heat. Heat emission also requires one hour, during which time the character is easily visible in the thermal infrared spectrum. \textbf{[Low]}

\textbf{Covert Operations Tool (COT):} This handheld device is the ultimate in infiltration technology. It contains both smart matter micromanipulators, cutting tools, and an advanced nanotechnology generator capable of producing nanobots that can bore or cut through almost any material and disable or open almost any electronic lock.

Cutting out a lock or boring a 1-millimeter hole in a wall with a COT requires ((Durability + Armor) $\div$ 10) seconds. Cutting out a 1-meter diameter hole in a wall requires ((Durability + Armor) $\div$ 10) minutes. These same nanobots can later be used to repair this damage so that it is invisible to any but the most careful and detailed examination.

A COT can easily open any old-fashioned mechanical lock simply by analyzing it and shaping an appropriate key, though this takes a full Action Turn. It can also open electronic locks by infiltrating them with nanobots that influence the lock’s electronics, no matter what authentication system the lock uses. Opening electronic locks takes a full Action Turn, but success is practically guaranteed. Opening an electronic lock in this manner will, however, trigger an alarm and/or be logged as an event. For more details, see \emph{Electronic Locks}, p. 291. \textbf{[High]}

\textbf{Cuffband:} This smart plastic loop restricts around a prisoner’s limbs when activated. If the prisoner struggles, it will tighten more. Cuffbands will inform the user if they are cut or loosened and are electronically- controlled, so the user can release the prisoner remotely. Some cuffband variants including a shock system (treat as a shock baton, p. 334) to zap and restrain unruly prisoners. \textbf{[Low]}

\textbf{Dazzler:} The dazzler is a tiny laser system set on a rotating ball. When activated, it consistently spins and emits laser pulses in all directions. These laser pulses are not dangerous, but they detect the lenses of camera systems (including specs, viewers, and bot/ synthmorph sensors) and repeatedly zap them with laser pulses of varying strength to overload and dazzle them. For as long as a dazzler is active, any camera system (visual, infrared, and ultraviolet) within line of sight and within 200 meters is blinded. \textbf{[Moderate]}

\textbf{Disabler:} This handy device emits an overloading surge that completely incapacitates and disables a synthetic morph or pod (anything with a cyberbrain) when it is plugged into an access jack and activated. The affected cyberbrain will be unable to function until the signal is deactivated, effectively shutting down the ego (or AI). In order to plug a disabler into an unwilling target, the target must first be grappled or a called shot must be successfully made in melee combat. This device does not work on larger synthetic morphs (like vehicles) or on cyberbrainless robots. \textbf{[High]}

\textbf{Fiber Eye:} This is a flexible and electronically-controllable length of fiberoptic cable and viewer, which can be worked through cracks, under doors, and around corners to peep unobtrusively. \textbf{[Low]}

\textbf{Invisibility Cloak:} This cloak is made of metamaterials with a negative refractive index, so that light actually bends around it, making it and anything it covers invisible. This invisibility works from the microwave to ultraviolet spectrums, but not against radar or x-rays. The drawback is that anything concealed within the cloak can’t see out. This is easily overcome by using external sensor feeds (if available) and entoptics to navigate. Alternately, a small piece of anti-cloak, which cancels the cloak’s invisibility properties when touched together, can be used to create a small window to peep out of, though this increases the chance of being spotted. Noticing such a window requires a Perception Test with a $-$30 modifier. \textbf{[High]}

\textbf{Microbug:} This device is a tiny camera and microphone 1 millimeter across. It has the visual capabilities of a set of specs (p. 325). It can hear everything within 20 meters and see everything within the same range that is in its line of sight. A microbug can record up to 100 hours of information. Microbugs can be set to broadcast continuously, at set intervals, or only when they receive a special signal. If desired, they can also be set to only record if there is movement or voices in the room they are in. Microbugs have adhesive backs and can stick to almost any surface. Microbugs can also establish their location via mesh positioning or GPS, and so double as tracking devices. To avoid being detected by their radio transmissions, some microbugs are attached to miniature quantum farcasters (p. 314). These microbugs are much larger (1 centimeter) and easy to see, but their transmissions cannot be detected or blocked. \textbf{[Trivial, Low for quantum farcaster bugs]}

\textbf{Prisoner Mask: }This hood tightens around the head of a prisoner, blocks all vision frequencies, and engages in low-level jamming in order to prevent any wireless communication via mesh inserts. \textbf{[Medium]}

\textbf{Psi Jammer:} This device jams frequencies used by brainwaves within a 20-meter radius. This has no effect on brain functions, but it does prevent any ranged used of psi sleights within this area of effect. \textbf{[Moderate]}

\textbf{Quantum Computer:} These advanced devices make use of quantum computation, allowing them to handle extremely large numbers with ease. This makes them especially useful for codebreaking, as noted on p. 254. \textbf{[Expensive]}

\textbf{Smart Dust:} This device is a walnut-sized specialized nanobot generator that creates tiny sensor nanobots, each one of which is a tiny sphere the diameter of a human hair. A packet of smart dust nanobots is sufficient to perform detailed surveillance on a large room like an auditorium has a volume of 1 cubic centimeter and contains 3 million nanobots. Each nanobot contains tiny cameras, microphones, a tiny computer, a radio, and chemical sensors, as well as short legs that allow them to walk and climb at a rate of 5 cm per second.

When a character dumps a packet of smart dust in a room, it will cover every surface in the room within 20 minutes, including all furniture and the insides of every drawer and other space that is not airtight. At this point, the smart dust has recorded all data about the room that can be obtained by exceedingly detailed observation, including the DNA of everyone who has visited the room in the last week or two. The smart dust can then either broadcast a brief, highly compressed signal, or it can send all of its information to a few hundred nanobots that then walk to a pre-arranged destination for pickup and downloading by their user. The user need only find a single nanobot with a nanodetector to acquire the information obtained by the smart dust. If ordered to do so, the remaining nanobots can either power down and await further orders or self-destruct in a fashion that turns them into a tiny amount of dust made mostly of metal and silicon. \textbf{[Moderate]}

\textbf{Traction Pads:} This set of specialized fingerless gloves, shoes, and kneepads is designed to emulate the pads on geckos’ feet. Characters can support themselves on a wall or ceiling by placing any two of these pads against any surface not made from a material specially designed to resist such devices. Characters can climb any surface and move easily across walls and ceilings that can support their weight (+30 to Climbing Tests). In addition to climbing, these devices are also very popular in zero-g environments. Wearing this item does not impair the user’s agility or manual dexterity. \textbf{[Low]}

\textbf{White Noise Machine:} This small and wearable device generates masking sounds that protect a conversation from being audibly recorded or overheard by anyone not in the immediate vicinity. \textbf{[Trivial]}

\textbf{X-Ray Emitter:} This device is designed to be used with either the enhanced vision augmentation (p. 301) or specs (p. 325). It emits a focused beam of low-powered x-rays that allows the user of either device to both see and see through most objects using backscatter x-ray radiation (p. 303). This allows the character to literally see through walls and into containers, including ones made of metal. \textbf{[Low]}

\subsection{Bugs and surveilance}
\label{sec:bugs-surveilance}

Though surveillance technologies are pervasive and easy to come by in Eclipse Phase, secretly obtaining information on someone who wants to retain privacy can be quite difficult. Microbugs, smart dust, and similar recording devices that are all but invisible may be exceptionally easy to put into place, but once they begin actively transmitting, they are easy to to detect (see Wireless Scanning, p. 251). An eavesdropper may attempt to stealth the signal (see Stealthed Signals, p. 252), but this is not guaranteed to work. Once a signal is detected, locating the broadcasting device is usually just a matter of time (see \textit{Tracking}, p. 251).

Some recording devices attempt to avoid this problem by using miniature quantum farcasters (p. 314), but those are far larger and more difficult to hide. Often the most effective way to acquire discrete information is to plant a surveillance device, set to record but not transmit, and then retrieve it later. While doing this is often difficult and risky, the recording device never reveals its presence by broadcasting and so is more difficult to detect.

\section{Drugs, chemicals and toxins}
\label{sec:drugs-chemicals-toxins}

In \emph{Eclipse Phase}, the transhuman desire to enhance the body and mind --- especially with chemicals ---  merges right into humanity’s popular pastime of recreational substance abuse. Drugs of all kinds, whether they be chemical, nano-based, or electronic, are not only popular but widespread. While advances in biotechnology have eliminated many of the side effects that once plagued drug users, transhuman bodies remain complicated environments, and so side effects (especially with long-term use) are still a factor. Additionally, addiction is always a consideration for anyone who gets comfortable with popping the same pills too often, though there are also drugs for addiction of course.

Drug descriptions include benefits, side effects, noticeable signs that a person is using the drug, addictiveness, and effects from long-term use). Descriptions also include the drug’s Duration and its Addiction Modifier (see \emph{Addiction and Substance Abuse}).

\subsection{Substance rules}
\label{sec:substance-rules}

These rules explain how to handle drugs and toxins.

\subsubsection{Classification of substances}

Substances fall into four categories:

\textbf{Chemicals:} These are pharmacologically-active small chemical compounds (toxins, pharmaceuticals, chemical drugs) that have been produced by chemical synthesis, nanotech fabrication, or enzymatic biosynthesis in (transgenic) organisms. They include naturally- occurring drugs from known species of (exo-)flora and fauna, endotoxins produced by biological organisms, enhancements of endogenic substances (designer drugs), and de novo developments designed for a specific medical or recreational application. Chemical drugs affect only biological morphs and pods.

\textbf{Biologicals:} These include peptides, hormones, and biologically-based substances like biotoxins, bacteria, and viral organisms --- drugs devised or based on naturally-occurring endogenic biological substances. This category also includes infectious biological organisms that can produce drug-like effects, like virii and bacteria. Biologicals affect biomorphs and pods but not synthetic morphs or infomorphs.

\textbf{Nanodrugs:} These are temporary nanobot colonies programmed to create a certain effect. While nanobots are generally able to target or infect all morph types except infomorphs, exactly which morphs are affected usually depends on the pre-programmed effect (i.e., whether it targets a biological or mechanical mechanism).

\textbf{Electronic:} Electronic drugs include software and technology that affect the brain directly, such as manipulative XP programs or retro-tech like transcranial magnetic stimulation or cranial electrotherapy. It also includes narcoalgorithms --- programs that reproduce drug-like effects for AIs, infomorphs, and egos residing in cyberbrains.

\subsubsection{Application methods}

There are number of vectors by which a substance may be applied to a morph.

\textbf{Dermal (D):} This drug or toxin is absorbed via the skin (or exterior hull with some nanotoxins) as either a gas, liquid, or solid (e.g., paste). Slap patches and slap bands are commonly used, loaded with the chemical DMSO, which transfers the drug through the skin.

\textbf{Inhalation (INH):} This is a gas that is breathed into the lungs or snorted nasally. Used for inhalers, aerosols, powders, and gas grenades/seekers.

\textbf{Injected (INJ):} This liquid is applied via either an intramuscular or intravenous injection. Used for needles and piercing weapons.

\textbf{Oral (O):} This is a liquid or solid that is absorbed through the stomach or oral cavity (eating or drinking). Used with pills and liquids.

\subsubsection{Drug effects}

If a character is exposed to a drug via its method of application --- for example, they pop a pill, slap on a dermal patch, are soaked with a splash grenade, breathe in gas, or get stabbed with a coated weapon ---  then they are subject to the drug’s effects. The onset time determines how long these effects take to kick in, and the duration determines how long they last. While there is no resistance test to ignore a drug or toxin’s effects once exposed, in some cases (especially toxins) a test might be called for to determine the \emph{severity} of the effects.

Unless otherwise noted or specifically overridden, medichines (p. 308) will protect a character from drug/toxin effects (but not nanodrugs/nanotoxins). Enhancements like toxin filters (p. 305) may also impede a drug’s effect or provide complete resistance. If an antidote is taken in advance or before the effects kick in, the drug will not work.

\subsubsection{Addiction and substance abuse}

Some drugs are addictive, either physically (affecting the morph) or mentally (affecting the ego) --- and sometimes both. Every time a character uses the drug (or after an appropriate amount of use, as determined by the gamemaster), they must make a WIL x 3 Test to avoid addiction. Each drug has an Addiction Modifier that will modify this test.

Failure indicates that the character has become addicted --- they immediately acquire the Addiction negative trait (p. 148). Addiction is measured in three levels: Minor, Moderate, and Major. The severity determines how often an addicted character needs the drug and what the negative effects of not using the drug are.

An addicted character must continue to make WIL x 3 Tests as they use the drug, as determined by the gamemaster. Failure indicates the character’s addiction severity increases.

The negative effects from not using a drug end whenever the character does the drug again. Durability and Lucidity penalties are not damage, but temporary decreases to the character’s maximum values; the character immediately regains the lost Durability or Lucidity when they do the drug again.

Addiction is of indefinite duration. To clean up, the character must stay off the drug for 1 week for each level of addiction. Resisting this craving is difficult, and should at least require another WIL x 3 Test, modified by the drug’s Addiction modifier. Players and gamemasters are encouraged to roleplay an attempt to kick a habit. Each week the character is off the drug, the addiction drops by one level. When it reaches 0, the character is clean ... though there is always danger of a relapse.

Physical addictions do not carry over to a new morph if the character resleeves, but mental addictions do. If the character uploads and resleeves, the mental addictions persist, and the morph the character leaves behind remains physically addicted. This means that poor or unlucky characters may occasionally find themselves resleeved into a morph that has a physical addiction. In this case, the character is subject to the physical addictiveness of the drug but not the mental addiction, although if they break down and indulge in the drug, they may themself become physically addicted.

Characters who resleeve as infomorphs can remain mentally addicted to a substance despite no longer having a body. The market is always happy to provide, though; a wide variety of narcoalgorithms mirroring the effects of most of the drugs described below are available for infomorphs and AIs. For the infomorphported narcoalgorithm version of any physicallyonly addictive drug described below, consider the Addictiveness to be effectively physical. The character remains addicted as long as they are an infomorph, but they do not remain addicted if they sleeve into a physical morph.

\subsection{Drugs}
\label{sec:drugs}

The drugs described here are usually (but not always beneficial), and are typically taken intentionally. Drugs and chemicals used offensively are described under Chemicals and Toxins, both on p. 323. Note that the drugs here are just a representative sampling. There are thousands if not millions of drugs in circulation in Eclipse Phase --- gamemasters are encouraged to introduce their own, using these as guidelines.

\subsubsection{Cognitive drugs}

Nootropics and similar drugs are intended to boost the user’s mental faculties.

\textbf{Drive:} This nootropic speeds up left-right brain hemisphere communication, stimulates idea production, and improves concentration, with no usual side effects. Users receive a +5 bonus to COG while the drug lasts. \textbf{[Low]}

\textbf{Klar:} Klar boosts alertness and enhances clarity and perception. Users report a feeling of being ``elevated'' to a higher level. They receive +5 INT while the drug lasts. \textbf{[Low]}

\textbf{Neem:} Neem is a mnemonic drug that works by ``tagging'' experiences and mental input with a set of unique sensations that contribute to the formation of state-based memories. Neem gummy chews come in a variety of fruit flavors shaped like extinct old Earth animals. Neem gives characters a +20 bonus on COG Tests to recall information they learned while on Neem (see \emph{Memorizing and Remembering}, p. 176). The drawback to Neem is that memories they accumulate while under the drug’s influence have no emotional association. For example, a character who witnessed something horrible happening to a friend or who had a fight with a romantic partner while on Neem would feel no emotional connection whatsoever to what happened. \textbf{[Moderate]}

\subsubsection{Combat drugs}

Combat drugs are an easy way of evening the odds in a fight.

\textbf{BringIt:} In some respects more a social than a combat drug, BringIt stimulates massive bursts of aggression pheromones designed to make the user the center of attention in a fight. In combat, opponents within 3 meters of the character not already in unarmed or melee combat with another character must pass a WIL x 3 Test or attack the character using BringIt. The nature of airborne pheromones is imprecise, however, so if the character using BringIt is within 1 meter of another character hostile to the character affected, the affected character may opt to attack the proximate character instead of the BringIt user. Characters using this drug suffer a $-$20 modifier on social skill tests. \textbf{[Low]}

\textbf{Grin:} Grin is an effective opiate and pain suppressant. Users may ignore the $-$10 modifiers from 2 wounds (not cumulative with similar effects), and in fact may not even be aware they are injured. Grin users suffer from tunnel vision, however, and so suffer a $-$10 modifier on Perception Tests. \textbf{[Low]}

\textbf{Kick:} Kick is a strong stimulant that increases the user’s response time and puts them on edge. The character gains +10 REF and +1 Speed for the duration of the drug. Characters under the influence of Kick are twitchy, however, reacting in a jumpy, cat-like fashion to sudden or unexpected stimuli. At the gamemaster’s discretion, they must make a WIL x 2 Test or react without thinking towards unexpected noises or other surprises. Long-term users suffer $-$5 COO. \textbf{[Moderate]}

\textbf{MRDR:} MRDR is a straightforward and brutal combat drug. It increases pain tolerance, speed, and strength. The character receives +10 SOM, +1 Speed, +10 Durability, and may ignore the $-$10 modifier of one wound. Any damage incurred while under the effects of the drug is taken from the bonus Durability first. MRDR users are easily identifiable by the broken blood vessels in their eyes, tense posture, and visible tension in the muscles of the face, arms, and legs. Long-term users suffer $-$5 SOM. \textbf{[Low]}

\textbf{Phlo:} Phlo increases alertness and coordination, making the user more graceful and nimble in a fray. The character gains +5 COO and +10 on Perception Tests for the duration of the drug. Everything feels possible to a character on Phlo, and so they are vulnerable to being goaded into actions that might be foolish or dangerous (apply a $-$10 modifier to appropriate Social Skill Tests). \textbf{[Moderate]}

\subsubsection{Health drugs}

Pharma-foods that boost the consumer’s health and physical state are common.

\textbf{Bananas Furiosas:} This drug reverses some of the effects of de-ionizing radiation on the cells of the body. Although a pill form is available, it most commonly comes in large bunches of bright orange-red bananas. Bananas reduce the severity of a radiation dosage (gamemaster determines effect). \textbf{[Low]}

Comfurt: This tasty yogurt treat blocks stress hormones, stabilizes mood, and relieves anxiety, allowing them to ignore the effect of 1 trauma and temporarily boosting Lucidity by +5. Any stress suffered while the drug is in effect is taken from the bonus Lucidity first. Comfurt also provides a +10 bonus when resisting attempts to manipulate the user’s emotions. Excessive use of Comfurt can lead to chronic itchiness caused by histamine release. [Low]

\subsubsection{Recreational drugs}

These drugs compete with petals (p. 321) and black market XP for wasting people’s time and lives away.

\textbf{Buzz:} This gene-modified variant of BZ is an odorless, invisible, extremely powerful hallucinogen. Users or affected characters will undergo extremely realistic hallucinations for the duration, and may even ``share'' hallucinations with other affected characters. Characters will suffer a $-$30 modifier to any tests to remember what occurred while under the influence. \textbf{[Moderate]}

\textbf{Mono No Aware:} Taken from the Japanese term for sadness at the ephemerality of worldly things, this drug, typically ingested as a tea, is a depressant that induces a meditative state. Mono No Aware gives the character a +10 bonus on Art and Sense Tests. With frequent use, Mono No Aware reacts with pigments in the skin to create a pallor with a slight bluish tinge, even in darker-skinned morphs. \textbf{[Low]}

\textbf{Orbital Hash:} Good ol’ reefer --- but grown in space using powerful lighting and post-singularity hydroponics. Because space is at a premium in habitats and scum barges, blocks of hashish are the preferred mode of transport and delivery. However, for the wealthy and on planets, buds in leaf form are not uncommon. Hash allows the character to ignore the effects of 1 trauma, but inflicts a $-$10 penalty on all memory-related tests and Knowledge Skill Tests. Hash users exhibit bloodshot eyes, lethargic behaviors, and the munchies. \textbf{[Low]}

\subsubsection{Social drugs}

These social lubricants affect the user’s interactions with others.

\textbf{Alpha:} Alpha is a more subtle version of BringIt, popular with hypercorp execs, street thugs, and anyone else who wants to come across as a domineering asshole. The pharm designer who invented it had a retro sensibility (and maybe a sick sense of humor); Alpha is typically synthesized as a sparkling white powder designed to be snorted. Alpha stimulates production of threat pheromones, but less bluntly than BringIt. Alpha imparts confidence, a feeling of power, and alertness. Users can function without sleep for 4 days, after which point they need to catch up with at least 4 hours of sleep (remember morphs with basic biomods require less sleep). Dosed characters receive a +20 modifier on Intimidation Tests and +10 on Persuasion and Networking Tests where attitude is a factor (gamemaster discretion). These bonuses only apply to characters within 2 meters of the Alpha user.

On the downside, alpha users are impatient, unfocused assholes. At the gamemaster’s discretion, Social skill modifiers may be reversed to penalties with certain types of people. Additionally, Alpha users suffer $-$10 on all COG skill tests related to memory and coherent or logical thinking. Long-term users may suffer the COG penalty even when not on the drug; on it, they may be worse. \textbf{[High]}

\textbf{Hither:} Want to ooze sexy like a pleasure morph on a hot tin roof? For those desiring that slinky je-nesais- quoi, Hither is the tool. Hither is a clear, slippery gel, sometimes with a faint, musky, floral scent. Hither is applied to parts of the body with large concentrations of sweat glands, where the skin quickly absorbs it. Hither is a mild euphoriant, imparting a feeling of confidence and you-know-you-want-it-ness to the user. It also stimulates abundant production of lust pheromones. The character gains a +10 bonus on Persuasion Tests against targets who are possible to seduce. At the gamemaster’s discretion, this extends to Deception, Impersonate, and Networking Tests. \textbf{[Low]}
\textbf{Juice:} This potent anti-depressant makes it almost impossible to have bad feelings or negative thoughts. The character is unnaturally happy --- often irritatingly or strangely so. The character receives a +30 bonus against fear or attempts to manipulate their emotions in a negative direction, but is also likely to act inappropriately, like giggling over the massive amount of spilled blood or cheerfully changing the subject to inane topics when someone else is freaking out. \textbf{[Low]}

\begin{table}
\begin{tabularx}{\textwidth}{|X|l|l|l|X|X|X|}
\hline
\multicolumn{7}{|l|}{\textbf{Drugs}}			\\
\hline
 & 				\textbf{Type}	& \textbf{Application}	& \textbf{Onset}	& \textbf{Duration}	& \textbf{Addiction\newline mod}	& \textbf{Addiction\newline type} \\
\hline
\multicolumn{7}{|l|}{\emph{Cognitive Drugs}}			\\
\hline
Drive			& Chem		& O					& 20 min 			& 8 hours			& $-$					& Mental	\\
\hline
Klar				& Chem		& O					& 20 min			& 8 hours			& $-$					& Mental	\\
\hline
Neem				& Chem		& O					& 20 min			& 12 hours		& $-$					& Mental	\\
\hline
\multicolumn{7}{|l|}{\emph{Combat Drugs}}		\\
\hline
BringIt			& Bio		& Inh, Inj, O			& 1 min			& 15 min			& +10					& Physical	\\
\hline
Grin				& Chem		& Inh, Inj, O			& 3 turns			& 3 hours			& $-$10					& Physical	\\
\hline
Kick				& Chem		& Inh, Inj, O			& 3 turns			& 2 hours			& $-$10					& Physical	\\
\hline
MRDR				& Chem		& O					& 20 min			& 1 hour			& $-$10					& Physical	\\
\hline
Phlo				& Chem		& O					& 20 min			& 1 hour			& $-$10					& Physical	\\
\hline
\multicolumn{7}{|l|}{\emph{Health Drugs}}		\\
\hline
Bananas Furiosas	& Chem		& O					& 20 min			& 1 day			& $-$					& $-$	\\
\hline
Comfurt			& Bio		& O					& 20 min			& 12 hours		& $-$10					& Mental	\\
\hline
\multicolumn{7}{|l|}{\emph{Recreational Drugs}}			\\
\hline
Buzz				& Chem		& Inh, O				& 1 hour			& 36 hours		& $-$					& Mental	\\
\hline
Mono No Aware		& Chem		& O					& 20 min			& 8 hours			& $-$10					& Mental	\\
\hline
Orbital Hash		& Chem		& Inh				& 3 min			& 3 hours			& $-$					& Mental	\\
\hline
\multicolumn{7}{|l|}{\emph{Social Drugs}}			\\
\hline
Alpha			& Bio		& Inh				& 1 minute		& 2 hours			& $-$10					& Mental	\\
\hline
Hither			& Bio		& D					& 1 minute		& 6 hours			& $-$10					& Physical	\\
\hline
Juice			& Chem		& O, Inh				& 20 min			& 8 hours			& $-$					& Mental	\\
\hline
\end{tabularx}
\label{tab:drugs}
\end{table}


\subsection{Nanodrugs}
\label{sec:nanodrugs}

\subsubsection{Petals}

\subsubsection{Sample petals}

\subsubsection{Other nanodrugs}


\subsection{Sample petals}
\label{sec:sample-petals}

\subsection{Narcoalgorithms}
\label{sec:narcoalgorithms}

\subsection{Chemicals}
\label{sec:chemicals}

\subsection{Toxins}
\label{sec:toxins}

\subsection{Nanotoxins}
\label{sec:nanotoxins}

\subsection{Pathogens}
\label{sec:pathogens}

\subsection{Psi drugs}
\label{sec:psi-drugs}

\section{Everyday technology}
\label{sec:everyday-tech}

\section{Nanotechnology}
\label{sec:nanotech}

\subsection{Basic nanotechnology}
\label{sec:basic-nanotech}

\subsubsection{Healing vats}

\subsubsection{Nanodetectors}

\subsubsection{Nanofabricators}

\subsection{Advanced nanotechnology}
\label{sec:advanced-nanotech}

\subsubsection{Ego bridges}

\subsubsection{Nanoswarms and microswarms}

\subsection{Pets}
\label{sec:pets}

\subsection{Scavenger tech}
\label{sec:scavenger-tech}

\subsection{Services}
\label{sec:services}

\subsection{Software}
\label{sec:software}

\subsection{Survival gear}
\label{sec:survival-gear}

\section{Weapons}
\label{sec:weapons}

\subsection{Malee weapons}
\label{sec:melee-weapons}

\subsection{Kinetic weapons}
\label{sec:kinetic-weapons}

\subsection{Brand name weapons and combined arms}
\label{sec:brand-weapons-combined}

\subsection{Beam weapons}
\label{sec:beam-weapons}

\subsection{Seekers}
\label{sec:seekers}

\subsection{Spray weapons}
\label{sec:spray-weapons}

\subsection{Grenades and seekers}
\label{sec:grenades-seekers}

\subsection{Exotic ranged weapons}
\label{sec:exotic-ranged-weapons}

\subsection{Weapon accessories}
\label{sec:weapon-accessories}

\section{Robots and vehicles}
\label{sec:robots-vehicles}

\subsection{Aircraft}
\label{sec:aircraft}

\subsection{Exoskeletons}
\label{sec:exoskeletons}

\subsection{Groundcraft}
\label{sec:groundcraft}

\subsection{Personal vehicles}
\label{sec:personal-vehicles}

\subsection{Robots}
\label{sec:robots}

\subsection{Spacecraft}
\label{sec:spacecraft}

%%% Local Variables: %%% mode: latex %%% TeX-master: "ep" %%% End: 
